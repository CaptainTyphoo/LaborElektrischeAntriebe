\documentclass[11pt,a4paper]{scrartcl}
\usepackage{a4wide}
\usepackage{fancyhdr}
\usepackage[naustrian]{babel}
\usepackage[utf8]{inputenc}
\usepackage{enumerate}%Aufzählungen
\usepackage{amsmath}%Formeln
\usepackage[locale=DE]{siunitx}%Einheiten
\usepackage{eurosym}%Eurosymbol
\usepackage{tikz}%Zeichnungen
\usepackage{pgfplots}%Funktionen plotten
\pgfplotsset{compat=1.10}
\usepackage[european]{circuitikz}%Schaltungen
\usetikzlibrary{decorations.pathreplacing,arrows}
\usepackage{tabularx}%Tabellen
\usepackage{trfsigns}%Korrespondezsymbole
\usepackage{ulem}%Unterstreichen von Text
\allowdisplaybreaks%Seitenumbruch in align Umgebung erlauben
\usepackage{titlesec}%neue Überschriften definieren
\usepackage{subcaption}%mehrere Grafiken nebeneinander darstellen
\usepackage{booktabs}
\usepackage{hyperref}%klickbares Inhaltsverzeichnis


\pagestyle{fancy} %eigener Seitenstil
\fancyhf{} %alle Kopf- und Fußzeilenfelder bereinigen
\fancyhead[L]{Labor Elektrische Antriebe\\UE 370.042} %Kopfzeile links
\fancyhead[C]{Lösungen zu Test Beispielen} %zentrierte Kopfzeile
\fancyhead[R]{v2.0\\www.fet.at} %Kopfzeile rechts
\renewcommand{\headrulewidth}{0.4pt} %obere Trennlinie
\fancyfoot[C]{\thepage} %Seitennummer
\renewcommand{\footrulewidth}{0.4pt} %untere Trennlinie
\setlength{\headheight}{26pt} %Höhe der Kopfzeile

\sisetup{
  per-mode = fraction,
}
\DeclareSIUnit\rounds{U}

\newcommand{\mybr}[1]{\left(#1\right)}
\newcommand{\ugamma}{\underline{\gamma}}
\renewcommand{\j}{\mathrm{j}}
\newcommand{\Z}{\underline{Z}}
\newcommand{\z}{\underline{z}}
\newcommand{\y}{\underline{y}}
\renewcommand{\S}{\underline{S}}
\renewcommand{\u}{\underline{u}}
\newcommand{\U}{\underline{U}}
\newcommand{\I}{\underline{I}}
\renewcommand{\i}{\underline{i}}
\newcommand{\E}{\underline{E}}
\newcommand{\PPsi}{\underline{\Psi}}
\newcommand{\0}{_{\mybr{0}}}
\newcommand{\1}{_{\mybr{1}}}
\newcommand{\2}{_{\mybr{2}}}
\newcommand{\UPS}{U_{1,Str}}
\newcommand{\USS}{U_{2,Str}}
\newcommand{\UPA}{U_{1,AL}}
\newcommand{\USA}{U_{2,AL}}
\newcommand{\IPS}{I_{1,Str}}
\newcommand{\ISS}{I_{2,Str}}
\newcommand{\IPA}{I_{1,AL}}
\newcommand{\ISA}{I_{2,AL}}
\newcommand{\UPNS}{U_{1,N,Str}}
\newcommand{\USNS}{U_{2,N,Str}}
\newcommand{\UPNA}{U_{1,N,AL}}
\newcommand{\USNA}{U_{2,N,AL}}
\newcommand{\IPNS}{I_{1,N,Str}}
\newcommand{\ISNS}{I_{2,N,Str}}
\newcommand{\IPNA}{I_{1,N,AL}}
\newcommand{\ISNA}{I_{2,N,AL}}
\newcommand{\ce}{\cos\mybr{\varphi_1}}
\newcommand{\se}{\sin\mybr{\varphi_1}}
\newcommand{\cz}{\cos\mybr{\varphi_2}}
\newcommand{\sz}{\sin\mybr{\varphi_2}}
\renewcommand{\a}{\underline{a}}
\renewcommand{\e}{\mathrm{e}}
\renewcommand{\d}{\mathrm{d}}
\renewcommand{\Re}{\mathrm{Re}}
\renewcommand{\Im}{\mathrm{Im}}
\newcommand{\isab}{\i_{S,\alpha\beta}}
\newcommand{\isdq}{\i_{S,dq}}
\newcommand{\usab}{\u_{S,\alpha\beta}}
\newcommand{\usdq}{\u_{S,dq}}
\newcommand{\psab}{\PPsi_{S,\alpha\beta}}
\newcommand{\psdq}{\PPsi_{S,dq}}

%Überschrift ohne Numerierung
\makeatletter
\@addtoreset{section}{part}
\makeatother
\titleformat{\part}[display]
{\normalfont\LARGE\bfseries\centering}{}{0pt}{}

%Inhaltsverzeichnis
\setcounter{tocdepth}{0}


\begin{document}
\tableofcontents
\newpage
%\iffalse
%%%%%%%%%%%%%%%%%%%%%%%%%%%%%%%%%%%%%%%%%%%%%%%%%%%%%%%%%%%%%%%%%%%%%%%%%%%%%%%%%%%%%%%%%%%%%%%%%%%%
%%%%%%%%%%%%%%%%%%%%%%%%%%%%%%%%%%%%%%%%%%%%%%%%%%%%%%%%%%%%%%%%%%%%%%%%%%%%%%%%%%%%%%%%%%%%%%%%%%%%
\part{2007 Trafo}
\section{und 2}
\begin{figure*}[!h]
\begin{subfigure}{.5\textwidth}
\centering
\begin{tikzpicture}[>=triangle 45,thick,node distance=0.5cm]

\path (0,0) coordinate (origin1);
\path (90:3cm) coordinate (1U);
\path (2*120+90:3cm) coordinate (1V);
\path (1*120+90:3cm) coordinate (1W);
\path (0,-6cm) coordinate (origin2);
\path (origin2) ++(90+30:3cm) coordinate (2U);
\path (origin2) ++(90+30+2*120:3cm) coordinate (2V);
\path (origin2) ++(90+30+1*120:3cm) coordinate (2W);
\path (1.5cm,-0.3cm) coordinate (U1V1N);
\path (origin2) ++(1.2cm,1.8cm) coordinate (U2U2V);

\path (240:3cm) coordinate (Usek);
\path (240:1cm) coordinate (arc1);
\path (240:2.5cm) coordinate (arc2);
\path (90:1.3cm) coordinate (arc3);
\path (-0.3cm, 1.3cm) coordinate (arc3help);

\draw [->] (origin1) -- (1U);
\draw [->] (origin1) -- (1V);
\draw [->] (origin1) -- (1W);

\draw [->] (origin2) -- (2U);
\draw [->] (origin2) -- (2V);
\draw [->] (origin2) -- (2W);

\draw [->] (2V) -- (2U);
\draw [->] (2U) -- (2W);
\draw [->] (2W) -- (2V);

\node [above of =1U] {1U};
\node [below right of =1V] {1V};
\node [below left of =1W] {1W};
\node [above of =2U] {2U};
\node [right of =2V] {2V};
\node [below left of =2W] {2W};
\node [right of =U1V1N] {$U_{1V1N}=U_b$};
\node [right of =U2U2V] {$U_{2U2V}=-U_e$};

\end{tikzpicture}
\end{subfigure}%
\begin{subfigure}{.5\textwidth}
\centering
\begin{circuitikz}

%\draw [help lines] (-1,-1) grid (12,17); %Zeichnet Raster und vereinfacht damit das Zeichnen
	
	%U
	\draw (0,0) node[above=5mm] {$1U$}
	to[american inductor, o-] (0,-3);
	\draw (0,-0.7) node[right=1.5mm] {$\bullet$};

	\draw (0,-5) node[above=5mm] {$2U$}
	to[american inductor, o-] (0,-9);
	\draw (0,-6.2) node[right=1.5mm] {$\bullet$};

	%V
	\draw (2,0) node[above=5mm] {$1V$}
	to[american inductor, o-] (2,-3);
	\draw (2,-0.7) node[right=1.5mm] {$\bullet$};

	\draw (2,-5) node[above=5mm] {$2V$}
	to[american inductor, o-] (2,-9);
	\draw (2,-6.2) node[right=1.5mm] {$\bullet$};
	
	%V
	\draw (4,0) node[above=5mm] {$1W$}
	to[american inductor, o-] (4,-3);
	\draw (4,-0.7) node[right=1.5mm] {$\bullet$};

	\draw (4,-5) node[above=5mm] {$2W$}
	to[american inductor, o-] (4,-9);
	\draw (4,-6.2) node[right=1.5mm] {$\bullet$};
	
	%Verbindung
	\draw (0,-3) to[short,-*] (2,-3)
	-- (4,-3);
	\draw (0,-5.5) to[short,*-] (1,-5.5)
	-- (1,-9)
	-- (2,-9);
	\draw (2,-5.5) to[short,*-] (3,-5.5)
	-- (3,-9)
	-- (4,-9);
	\draw (4,-5.5) to[short,*-] (5,-5.5)
	-- (5,-10)
	-- (0,-10)
	-- (0,-9);

	%Spannungspfeile
	\draw (-0.2,0) to [open, v>=$U_a$] (-0.2,-3);
	\draw (1.8,0) to [open, v>=$U_b$] (1.8,-3);
	\draw (3.8,0) to [open, v>=$U_c$] (3.8,-3);

	\draw (-0.2,-5.5) to [open, v>=$U_d$] (-0.2,-8.5);
	\draw (1.8,-5.5) to [open, v>=$U_e$] (1.8,-8.5);
	\draw (3.8,-5.5) to [open, v>=$U_f$] (3.8,-8.5);
	
\end{circuitikz}
\end{subfigure}%
\end{figure*}

\stepcounter{section}
\section{}
\begin{equation}
\frac{N_1}{N_2}=\frac{\UPNS}{\USNS}=\frac{\frac{\SI{24}{\kilo\volt}}{\sqrt{3}}}{\SI{10.5}{\kilo\volt}}=\num{1.3197}
\end{equation}

\section{}
\begin{align}
\IPNS&=\frac{S_N}{3\UPNS}=\frac{\SI{35}{\mega\volt\ampere}}{3\frac{\SI{24}{\kilo\volt}}{\sqrt{3}}}=\SI{842.0}{\ampere}\\
\IPNA&=\IPNS=\SI{842.0}{\ampere}\\
\ISNS&=\frac{S_N}{3\USNS}=\frac{\SI{35}{\mega\volt\ampere}}{3\cdot\SI{10.5}{\kilo\volt}}=\SI{1111}{\ampere}\\
\ISNA&=\sqrt{3}\ISNS=\sqrt{3}\cdot\SI{1111}{\ampere}=\SI{1924}{\ampere}
\end{align}

\section{}
\subsection{}
\begin{equation}
u_1=\num{0.95}\quad i=\num{0.85}\quad \cos\mybr{\varphi_2}=0.9
\end{equation}
\begin{equation}
\sin\mybr{\varphi_2}=-\sqrt{1-\cz^2}=-\sqrt{1-\num{0.9}^2}=\num{-0.4359}
\end{equation}
\begin{align}
u_1\ce&=u_2\cz+u_R i\\
u_1\se&=u_2\sz+u_X i\\
u_1^2&=\mybr{u_1\ce}^2+\mybr{u_1\se}^2\\
&=\mybr{u_2\cz+u_R i}^2+\mybr{u_2\sz+u_X i}^2\\
&=u_R^2i^2+u_X^2i^2+u_2^2+2u_R i u_2 \cz+2u_X i u_2\sz
\end{align}
\begin{align}
u_2^2+u_2\mybr{2u_R i \cz+2u_X i \sz}+u_R^2 i^2 + u_X^2 i^2 - u_1^2 &= 0\\
u_2^2+u_2\mybr{2\cdot\num{0.013}\cdot\num{0.85}\cdot\num{0.9}+2\cdot\num{0.084}\cdot\num{0.85}\cdot\mybr{\num{-0.4359}}}+\num{0.013}^2\cdot\num{0.85}^2&+\\
\num{0.084}^2\cdot\num{0.85}^2-\num{0.95}^2 &=0\\
u_2^2-\num{0.04236}u_2-\num{0.8973}&=0
\end{align}
\begin{align}
u_2&=\frac{\num{0.04236}\pm\sqrt{\num{0.04236}^2-4\cdot1\cdot\mybr{\num{-0.8973}}}}{2\cdot1}\\
u_{2,1}&=\num{0.9687}\\
u_{2,2}&=\num{-0.9263}
\end{align}
$u_2$ ist ein Betrag und kann somit nicht negativ sein, daher ist die Lösung $u_{2,1}$ richtig.
\begin{align}
\USS&=u_2\USNS=\num{0.9687}\cdot\SI{10.5}{\kilo\volt}=\SI{10.17}{\kilo\volt}\\
\USA&=u_2\USNA=\num{0.9687}\cdot\SI{10.5}{\kilo\volt}=\SI{10.17}{\kilo\volt}
\end{align}

\subsection{}
\begin{align}
\ISS&=i\ISNS=\num{0.85}\cdot\SI{1111}{\ampere}=\SI{944.4}{\ampere}\\
\ISA&=i\ISNA=\num{0.85}\cdot\SI{1924}{\ampere}=\SI{1635}{\ampere}\\
\end{align}

\subsection{}
\begin{equation}
P_2=S_N u_2 i \cz=\SI{35}{\mega\volt\ampere}\cdot\num{0.9687}\cdot\num{0.85}\cdot\num{0.9}=\SI{25.94}{\mega\watt}
\end{equation}

\subsection{}
\begin{equation}
Q_2=S_N u_2 i \sz=\SI{35}{\mega\volt\ampere}\cdot\num{0.9071}\cdot\num{0.85}\cdot\num{0.4359}=\SI{11.76}{\mega\volt\ampere}
\end{equation}

\clearpage
%%%%%%%%%%%%%%%%%%%%%%%%%%%%%%%%%%%%%%%%%%%%%%%%%%%%%%%%%%%%%%%%%%%%%%%%%%%%%%%%%%%%%%%%%%%%%%%%%%%%
%%%%%%%%%%%%%%%%%%%%%%%%%%%%%%%%%%%%%%%%%%%%%%%%%%%%%%%%%%%%%%%%%%%%%%%%%%%%%%%%%%%%%%%%%%%%%%%%%%%%
\part{2008 Trafo}
\section{}
\begin{equation}
\frac{N_1}{N_2}=\frac{\UPNS}{\USNS}=\frac{\frac{\SI{132}{\kilo\volt}}{\sqrt{3}}}{\frac{\SI{15}{\kilo\volt}}{\sqrt{3}}}=\num{8.8}
\end{equation}

\section{}
\begin{align}
\IPNS&=\frac{S_N}{3\UPNS}=\frac{\SI{80}{\mega\volt\ampere}}{3\cdot\frac{\SI{132}{\kilo\volt}}{\sqrt{3}}}=\SI{349.9}{\ampere}\\
\IPNA&=\IPNS=\SI{349.9}{\ampere}\\
\ISNS&=\frac{S_N}{3\USNS}=\frac{\SI{80}{\mega\volt\ampere}}{3\cdot\frac{\SI{15}{\kilo\volt}}{\sqrt{3}}}=\SI{3079}{\ampere}\\
\ISNA&=\ISNS=\SI{3079}{\ampere}
\end{align}

\section{}
\subsection{}
\begin{equation}
u_1=1\quad i=1\quad u_2=1
\end{equation}
\begin{align}
u_1\ce&=u_2\cz+u_R i\\
u_1\se&=u_2\sz+u_X i\\
u_1^2&=\mybr{u_1\ce}^2+\mybr{u_1\se}^2\\
&=\mybr{u_2\cz+u_R i}^2+\mybr{u_2\sz+u_X i}^2\\
&=u_R^2i^2+u_X^2i^2+u_2^2+2u_R i u_2 \cz+2u_X i u_2\sz\\
1&=u_X^2+u_2^2+2u_X\sz\\
0&=2\sz u_X+u_X^2\\
\sz&=-\frac{u_X}{2}=-\frac{\num{0.09}}{2}=\num{-0.045}\\
\cz&=\sqrt{1-\sz^2}=\sqrt{1-\num{0.045}^2}=\num{0.9990}
\end{align}
\begin{align}
\z_L&=\frac{u_2}{i}\e^{\j\varphi_2}=\cz+\j\sz=\num{0.9990}-\j\num{0.045}\\
\Z_L&=\z_L\frac{3\USNA^2}{S_N}=\mybr{\num{0.9990}-\j\num{0.045}}\frac{3\cdot\mybr{\SI{15}{\kilo\volt}}^2}{\SI{80}{\mega\volt\ampere}}=\SI{8.429}{\ohm}+\j\SI{0.3797}{\ohm}=R'+\j X'
\end{align}
Hinweis: Verwendet man $\frac{3U_S^2}{S_N}$ als Normierung für die der Lastimpedanzen, mit der an einem Strang der Lastimpedanz $\Z_L$ anliegenden Spannung $U_S$, erhält man immer die korrekte bezogene Impedanz bzw. wie hier, im umgekehrten Fall, die Impedanz eines Stranges der Last. In diesem Beispiel ist die Strangspannung an der Last gleich der Außenleiterspannung der sekundärseitigen Sternschaltung des Trafos.
\begin{align}
\frac{R\cdot\frac{1}{\j\omega C}}{R+\frac{1}{\j\omega C}}&=\frac{R}{1+\j\omega R C}=R'+\j X'\\
R&=\mybr{R'+\j\omega C'}\mybr{1+\j\omega R C}\\
\text{Realteil:}\\
R&=R'-X'\omega R C\\
\text{Imaginärteil:}\\
0&=X'+R'\omega R C\\
\omega R C &= -\frac{X'}{R'}\\
R &= R' + \frac{X'^2}{R'}=\SI{8.429}{\ohm} + \frac{\mybr{\SI{0.3797}{\ohm}}^2}{\SI{8.429}{\ohm}}=\SI{8.446}{\ohm}\\
C &= -\frac{X'}{R'\omega R}=\frac{\SI{0.3797}{\ohm}}{\SI{8.429}{\ohm}\cdot2\pi\SI{60}{\hertz}\cdot\SI{8.446}{\ohm}}=\SI{14.15}{\micro\farad}
\end{align}

\subsection{}
\begin{equation}
P_2=S_N u_2 i \cz=\SI{80}{\mega\volt\ampere}\cdot\num{1}\cdot\num{1}\cdot\num{0.9990}=\SI{79.92}{\mega\watt}
\end{equation}

\section{}
\begin{align}
I_{2K}&=\frac{\ISNA}{\left| u_R+\j u_X +\j x_{Netz}\right|}=\frac{\ISNA}{\left| u_R+\j u_X +\j X_{Netz}\frac{S_N}{3U_S^2}\right|}\\
&=\frac{\SI{3079}{\ampere}}{\left| \num{0}+\j\num{0.09}+\j\SI{4.598}{\ohm}\frac{\SI{80}{\mega\volt\ampere}}{3\cdot\mybr{\frac{\SI{132}{\kilo\volt}}{\sqrt{3}}}^2}\right|}=\SI{27.71}{\kilo\ampere}
\end{align}

\section{}
gleiche magnetische Beanspruchung:
\begin{equation}
\hat\Phi=\text{const.}
\end{equation}
aus
\begin{equation}
U=\frac{1}{\sqrt{2}}N\omega\hat\Phi
\end{equation}
folgt
\begin{align}
U_{1N,E}&=\frac{f_{N,E}}{f_{N}}U_{1N}=\frac{\SI{50}{\hertz}}{\SI{60}{\hertz}}\SI{132}{\kilo\volt}=\SI{110}{\kilo\volt}\\
U_{2N,E}&=\frac{f_{N,E}}{f_{N}}U_{2N}=\frac{\SI{50}{\hertz}}{\SI{60}{\hertz}}\SI{15}{\kilo\volt}=\SI{12.5}{\kilo\volt}
\end{align}
gleiche thermische Beanspruchung:
\begin{equation}
R_k I^2 = \text{const.}
\end{equation}
da sich R mit der Frequenz nicht ändert bleibt der Strom gleich
\begin{align}
I_{1N,E}&=I_{1N}=\SI{349.9}{\ampere}\\
I_{2N,E}&=I_{2N}=\SI{3079}{\ampere}
\end{align}
\begin{equation}
S_N=3U_{1N,E,Str}I_{1N,E}=3\cdot\frac{\SI{110}{\kilo\volt}}{\sqrt{3}}\cdot\SI{349.9}{\ampere}=\SI{66.66}{\mega\volt\ampere}
\end{equation}

\section{}
\begin{align}
u_{R,E}&=R_k\frac{S_{N,E}}{3U_{1N,E,Str}^2}=u_R\frac{3\UPNS^2}{S_N}\frac{S_{N,E}}{3U_{1N,E,Str}^2}=u_R\frac{\frac{5}{6}}{\mybr{\frac{5}{6}}^2}=\num{0.0}\cdot\frac{6}{5}=\num{0.0}\\
u_{X,E}&=L_k\frac{S_{N,E}\omega_{N,E}}{3U_{1N,E,Str}^2}=u_X\frac{3\UPNS^2}{S_N\omega_N}\frac{S_{N,E}\omega_{N,E}}{3U_{1N,E,Str}^2}=\\
&=u_X\frac{\mybr{\frac{5}{6}}^2}{\mybr{\frac{5}{6}}^2}=u_X=\num{0.09}\\
u_{K,E}&=\sqrt{u_{R,E}^2+u_{X,E}^2}=0.09
\end{align}

\clearpage
%%%%%%%%%%%%%%%%%%%%%%%%%%%%%%%%%%%%%%%%%%%%%%%%%%%%%%%%%%%%%%%%%%%%%%%%%%%%%%%%%%%%%%%%%%%%%%%%%%%%
%%%%%%%%%%%%%%%%%%%%%%%%%%%%%%%%%%%%%%%%%%%%%%%%%%%%%%%%%%%%%%%%%%%%%%%%%%%%%%%%%%%%%%%%%%%%%%%%%%%%
\part{2009 Trafo}
\section{}
\begin{figure*}[!h]
\begin{subfigure}{.55\textwidth}
\centering
\begin{tikzpicture}[>=triangle 45,thick,node distance=0.5cm]

\path (0,0) coordinate (origin1);
\path (90:3cm) coordinate (1U);
\path (2*120+90:3cm) coordinate (1V);
\path (1*120+90:3cm) coordinate (1W);
\path (0,-6cm) coordinate (origin2);
\path (origin2) ++(90+150:3cm) coordinate (2U);
\path (origin2) ++(90+150+2*120:3cm) coordinate (2V);
\path (origin2) ++(90+150+1*120:3cm) coordinate (2W);
\path (0.7cm,1.7cm) coordinate (U1U1N);
\path (origin2) ++(-2.5cm,0cm) coordinate (U2U2V);

\path (240:3cm) coordinate (Usek);
\path (240:1cm) coordinate (arc1);
\path (240:2.5cm) coordinate (arc2);
\path (90:1.3cm) coordinate (arc3);
\path (-0.3cm, 1.3cm) coordinate (arc3help);

\draw [->] (origin1) -- (1U);
\draw [->] (origin1) -- (1V);
\draw [->] (origin1) -- (1W);

\draw [->] (origin2) -- (2U);
\draw [->] (origin2) -- (2V);
\draw [->] (origin2) -- (2W);

\draw [->] (2V) -- (2U);
\draw [->] (2U) -- (2W);
\draw [->] (2W) -- (2V);

\node [above of =1U] {1U};
\node [below right of =1V] {1V};
\node [below left of =1W] {1W};
\node [below left of =2U] {2U};
\node [above right of =2V] {2V};
\node [right of =2W] {2W};
\node [right of =U1U1N] {$U_{1U1N}=U_a$};
\node [left of =U2U2V] {$U_{2U2V}=-U_d$};

\end{tikzpicture}
\end{subfigure}%
\begin{subfigure}{.45\textwidth}
\centering
\begin{circuitikz}

%\draw [help lines] (-1,-1) grid (6,-10); %Zeichnet Raster und vereinfacht damit das Zeichnen
	
	%U
	\draw (0,0) node[above=5mm] {$1U$}
	to[american inductor, o-] (0,-3);
	\draw (0,-0.7) node[right=1.5mm] {$\bullet$};

	\draw (0,-5) [american inductor, -o] 
	to node[below=10mm] {$2U$} (0,-9);
	\draw (0,-6.2) node[right=1.5mm] {$\bullet$};

	%V
	\draw (2,0) node[above=5mm] {$1V$}
	to[american inductor, o-] (2,-3);
	\draw (2,-0.7) node[right=1.5mm] {$\bullet$};

	\draw (2,-5) [american inductor, -o]
	to node[below=10mm] {$2V$} (2,-9);
	\draw (2,-6.2) node[right=1.5mm] {$\bullet$};
	
	%V
	\draw (4,0) node[above=5mm] {$1W$}
	to[american inductor, o-] (4,-3);
	\draw (4,-0.7) node[right=1.5mm] {$\bullet$};

	\draw (4,-5) [american inductor, -o] 
	to node[below=10mm] {$2W$} (4,-9);
	\draw (4,-6.2) node[right=1.5mm] {$\bullet$};
	
	%Verbindung
	\draw (0,-3) to[short,-*] (2,-3)
	-- (4,-3);
	\draw (0,-5) -- (1,-5)
	-- (1,-8.5)
	to[short,-*] (2,-8.5);
	\draw (2,-5) -- (3,-5)
	-- (3,-8.5)
	to[short,-*] (4,-8.5);
	\draw (4,-5) -- (4,-4.5)
	-- (-1,-4.5)
	-- (-1,-8.5)
	to[short,-*] (0,-8.5);

	%Spannungspfeile
	\draw (-0.2,0) to [open, v>=$U_a$] (-0.2,-3);
	\draw (1.8,0) to [open, v>=$U_b$] (1.8,-3);
	\draw (3.8,0) to [open, v>=$U_c$] (3.8,-3);

	\draw (-0.2,-5.5) to [open, v>=$U_d$] (-0.2,-8.5);
	\draw (1.8,-5.5) to [open, v>=$U_e$] (1.8,-8.5);
	\draw (3.8,-5.5) to [open, v>=$U_f$] (3.8,-8.5);
	
\end{circuitikz}
\end{subfigure}%
\end{figure*}
\begin{equation}
\frac{N_1}{N_2}=\frac{\UPNS}{\USNS}=\frac{\frac{\SI{240}{\kilo\volt}}{\sqrt{3}}}{\SI{21}{\kilo\volt}}=\num{11.55}
\end{equation}

\section{}
\begin{align}
\IPNS&=\frac{S_N}{3\UPNS}=\frac{\SI{850}{\mega\volt\ampere}}{3\frac{\SI{420}{\kilo\volt}}{\sqrt{3}}}=\SI{1168}{\ampere}\\
\IPNA&=\IPS=\SI{1168}{\ampere}\\
\ISNS&=\frac{S_N}{3\USNS}=\frac{\SI{850}{\mega\volt\ampere}}{3\cdot\SI{21}{\kilo\volt}}=\SI{13.49}{\kilo\ampere}\\
\ISNA&=\sqrt{3}\ISNS=\sqrt{3}\cdot\SI{13.49}{\kilo\ampere}=\SI{23.37}{\kilo\ampere}
\end{align}

\section{}
\subsection{}
\begin{equation}
u_1=1\quad i=1\quad \cz=0\quad \sz=-1
\end{equation}
\begin{align}
u_1\ce&=u_2\cz+u_R i\\
u_1\se&=u_2\sz+u_X i\\
\ce&=\cz\quad \rightarrow\quad \se = \pm 1\\
u_1\se&=1\cdot\mybr{-1}+1\cdot u_X\\
u_1\se&=\num{-0.875}\\
u_1&=\num{0.875}\quad\text{$u_1$ ist ein Betrag und somit $>0$}\\
\se&=\num{-1}\\
U_{20}&=u_1\USNA=\num{0.875}\cdot\SI{21}{\kilo\volt}=\SI{18.38}{\kilo\volt}
\end{align}

\subsection{}
\begin{align}
\ddot{u}&=\frac{U_{Netz}}{U_{20}}=\frac{U_{1,AL}}{\USNA}\\
&=\frac{\SI{385}{\kilo\volt}}{\SI{18.38}{\kilo\volt}}=\frac{\SI{420}{\kilo\volt}+x\cdot\SI{5}{\kilo\volt}}{\SI{21}{\kilo\volt}}\\
x&=\frac{\frac{\SI{385}{\kilo\volt}}{\SI{18.38}{\kilo\volt}}\cdot\SI{21}{\kilo\volt}-\SI{420}{\kilo\volt}}{\SI{5}{\kilo\volt}}=4
\end{align}

\section{}
\subsection{}
\begin{equation}
i=\num{0.9}\quad \cz=-0.8
\end{equation}
\begin{align}
u_1&=\frac{\SI{385}{\kilo\volt}}{\SI{420}{\kilo\volt}}=\num{0.9167}\\
\sz&=-\sqrt{1-\cz^2}=-\sqrt{1-{\num{0.8}}^2}=\num{-0.6}\\
u_1\ce&=u_2\cz+u_R i\\
u_1\se&=u_2\sz+u_X i\\
u_1^2&=\mybr{u_1\ce}^2+\mybr{u_1\se}^2\\
&=\mybr{u_2\cz+u_R i}^2+\mybr{u_2\sz+u_X i}^2\\
&=u_R^2i^2+u_X^2i^2+u_2^2+2u_R i u_2 \cz+2u_X i u_2\sz
\end{align}
\begin{align}
u_2^2+u_2\mybr{2u_R i \cz+2u_X i \sz}+u_R^2 i^2 + u_X^2 i^2 - u_1^2 &= 0\\
u_2^2+u_2\mybr{\num{0.0}+2\cdot\num{0.125}\cdot\num{0.9}\cdot\mybr{\num{-0.6}}}+\num{0.0}+\num{0.125}^2\cdot\num{0.9}^2-\num{0.9167}^2 &=0\\
u_2^2+\num{0.135}u_2-\num{0.8277}&=0
\end{align}
\begin{align}
u_2&=\frac{\num{0.135}\pm\sqrt{\num{0.135}^2-4\cdot1\cdot\mybr{\num{-0.8277}}}}{2\cdot1}\\
u_{2,1}&=\num{0.9798}\\
u_{2,2}&=\num{-0.8448}
\end{align}
$u_2$ ist ein Betrag und kann somit nicht negativ sein, daher ist die Lösung $u_{2,1}$ richtig.
\begin{align}
\USA&=u_2\USNA=\num{0.9798}\cdot\SI{21}{\kilo\volt}=\SI{20.58}{\kilo\volt}
\end{align}

\subsection{}
\begin{equation}
P_2=S_N u_2 i \cz=\SI{850}{\mega\volt\ampere}\cdot\num{0.9798}\cdot\num{0.9}\cdot\num{-0.8}=\SI{-599.6}{\mega\watt}
\end{equation}

\clearpage
%%%%%%%%%%%%%%%%%%%%%%%%%%%%%%%%%%%%%%%%%%%%%%%%%%%%%%%%%%%%%%%%%%%%%%%%%%%%%%%%%%%%%%%%%%%%%%%%%%%%
%%%%%%%%%%%%%%%%%%%%%%%%%%%%%%%%%%%%%%%%%%%%%%%%%%%%%%%%%%%%%%%%%%%%%%%%%%%%%%%%%%%%%%%%%%%%%%%%%%%%
\part{2010 Trafo}
\section{}
\begin{figure*}[!h]
\begin{subfigure}{.5\textwidth}
\centering
\begin{tikzpicture}[>=triangle 45,thick,node distance=0.5cm]

\path (0,0) coordinate (origin1);
\path (90:3cm) coordinate (1U);
\path (2*120+90:3cm) coordinate (1V);
\path (1*120+90:3cm) coordinate (1W);
\path (0,-6cm) coordinate (origin2);
\path (origin2) ++(90+30:3cm) coordinate (2U);
\path (origin2) ++(90+30+2*120:3cm) coordinate (2V);
\path (origin2) ++(90+30+1*120:3cm) coordinate (2W);
\path (1.5cm,-0.3cm) coordinate (U1V1N);
\path (origin2) ++(1.2cm,1.8cm) coordinate (U2U2V);

\path (240:3cm) coordinate (Usek);
\path (240:1cm) coordinate (arc1);
\path (240:2.5cm) coordinate (arc2);
\path (90:1.3cm) coordinate (arc3);
\path (-0.3cm, 1.3cm) coordinate (arc3help);

\draw [->] (origin1) -- (1U);
\draw [->] (origin1) -- (1V);
\draw [->] (origin1) -- (1W);

\draw [->] (origin2) -- (2U);
\draw [->] (origin2) -- (2V);
\draw [->] (origin2) -- (2W);

\draw [->] (2V) -- (2U);
\draw [->] (2U) -- (2W);
\draw [->] (2W) -- (2V);

\node [above of =1U] {1U};
\node [below right of =1V] {1V};
\node [below left of =1W] {1W};
\node [above of =2U] {2U};
\node [right of =2V] {2V};
\node [below left of =2W] {2W};
\node [right of =U1V1N] {$U_{1V1N}=U_b$};
\node [right of =U2U2V] {$U_{2U2V}=-U_e$};

\end{tikzpicture}
\end{subfigure}%
\begin{subfigure}{.5\textwidth}
\centering
\begin{circuitikz}

%\draw [help lines] (-1,-1) grid (12,17); %Zeichnet Raster und vereinfacht damit das Zeichnen
	
	%U
	\draw (0,0) node[above=5mm] {$1U$}
	to[american inductor, o-] (0,-3);
	\draw (0,-0.7) node[right=1.5mm] {$\bullet$};

	\draw (0,-5) node[above=5mm] {$2U$}
	to[american inductor, o-] (0,-9);
	\draw (0,-6.2) node[right=1.5mm] {$\bullet$};

	%V
	\draw (2,0) node[above=5mm] {$1V$}
	to[american inductor, o-] (2,-3);
	\draw (2,-0.7) node[right=1.5mm] {$\bullet$};

	\draw (2,-5) node[above=5mm] {$2V$}
	to[american inductor, o-] (2,-9);
	\draw (2,-6.2) node[right=1.5mm] {$\bullet$};
	
	%V
	\draw (4,0) node[above=5mm] {$1W$}
	to[american inductor, o-] (4,-3);
	\draw (4,-0.7) node[right=1.5mm] {$\bullet$};

	\draw (4,-5) node[above=5mm] {$2W$}
	to[american inductor, o-] (4,-9);
	\draw (4,-6.2) node[right=1.5mm] {$\bullet$};
	
	%Verbindung
	\draw (0,-3) to[short,-*] (2,-3)
	-- (4,-3);
	\draw (0,-5.5) to[short,*-] (1,-5.5)
	-- (1,-9)
	-- (2,-9);
	\draw (2,-5.5) to[short,*-] (3,-5.5)
	-- (3,-9)
	-- (4,-9);
	\draw (4,-5.5) to[short,*-] (5,-5.5)
	-- (5,-10)
	-- (0,-10)
	-- (0,-9);

	%Spannungspfeile
	\draw (-0.2,0) to [open, v>=$U_a$] (-0.2,-3);
	\draw (1.8,0) to [open, v>=$U_b$] (1.8,-3);
	\draw (3.8,0) to [open, v>=$U_c$] (3.8,-3);

	\draw (-0.2,-5.5) to [open, v>=$U_d$] (-0.2,-8.5);
	\draw (1.8,-5.5) to [open, v>=$U_e$] (1.8,-8.5);
	\draw (3.8,-5.5) to [open, v>=$U_f$] (3.8,-8.5);
	
\end{circuitikz}
\end{subfigure}%
\end{figure*}

\section{}
\begin{equation}
\frac{N_1}{N_2}=\frac{\UPNS}{\USNS}=\frac{\frac{\SI{24}{\kilo\volt}}{\sqrt{3}}}{\SI{10.5}{\kilo\volt}}=\num{1.3197}
\end{equation}

\section{}
\begin{align}
\IPNS&=\frac{S_N}{3\UPNS}=\frac{\SI{35}{\mega\volt\ampere}}{3\frac{\SI{24}{\kilo\volt}}{\sqrt{3}}}=\SI{842.0}{\ampere}\\
\IPNA&=\IPS=\SI{842.0}{\ampere}\\
\ISNS&=\frac{S_N}{3\USNS}=\frac{\SI{35}{\mega\volt\ampere}}{3\cdot\SI{10.5}{\kilo\volt}}=\SI{1111}{\ampere}\\
\ISNA&=\sqrt{3}\ISNS=\sqrt{3}\cdot\SI{1111}{\ampere}=\SI{1924}{\ampere}
\end{align}

\section{}
\subsection{}
\begin{equation}
u_1=\num{0.95}\quad i=\num{0.85}\quad \cos\mybr{\varphi_2}=0.9
\end{equation}
\begin{equation}
\sin\mybr{\varphi_2}=-\sqrt{1-\cz^2}=-\sqrt{1-\num{0.9}^2}=\num{-0.4359}
\end{equation}
\begin{align}
u_1\ce&=u_2\cz+u_R i\\
u_1\se&=u_2\sz+u_X i\\
u_1^2&=\mybr{u_1\ce}^2+\mybr{u_1\se}^2\\
&=\mybr{u_2\cz+u_R i}^2+\mybr{u_2\sz+u_X i}^2\\
&=u_R^2i^2+u_X^2i^2+u_2^2+2u_R i u_2 \cz+2u_X i u_2\sz
\end{align}
\begin{align}
u_2^2+u_2\mybr{2u_R i \cz+2u_X i \sz}+u_R^2 i^2 + u_X^2 i^2 - u_1^2 &= 0\\
u_2^2+u_2\mybr{2\cdot\num{0.013}\cdot\num{0.85}\cdot\num{0.9}+2\cdot\num{0.084}\cdot\num{0.85}\cdot\mybr{\num{-0.4359}}}+\num{0.013}^2\cdot\num{0.85}^2&+\\
\num{0.084}^2\cdot\num{0.85}^2-\num{0.95}^2 &=0\nonumber\\
u_2^2-\num{0.04236}u_2-\num{0.8973}&=0
\end{align}
\begin{align}
u_2&=\frac{\num{0.04236}\pm\sqrt{\num{0.04236}^2-4\cdot1\cdot\mybr{\num{-0.8973}}}}{2\cdot1}\\
u_{2,1}&=\num{0.9687}\\
u_{2,2}&=\num{-0.9263}
\end{align}
$u_2$ ist ein Betrag und kann somit nicht negativ sein, daher ist die Lösung $u_{2,1}$ richtig.
\begin{align}
\USS&=u_2\USNS=\num{0.9687}\cdot\SI{10.5}{\kilo\volt}=\SI{10.17}{\kilo\volt}\\
\USA&=u_2\USNA=\num{0.9687}\cdot\SI{10.5}{\kilo\volt}=\SI{10.17}{\kilo\volt}
\end{align}

\subsection{}
\begin{align}
\ISS&=i\ISNS=\num{0.85}\cdot\SI{1111}{\ampere}=\SI{944.4}{\ampere}\\
\ISA&=i\ISNA=\num{0.85}\cdot\SI{1924}{\ampere}=\SI{1635}{\ampere}\\
\end{align}

\subsection{}
\begin{equation}
P_2=S_N u_2 i \cz=\SI{35}{\mega\volt\ampere}\cdot\num{0.9687}\cdot\num{0.85}\cdot\num{0.9}=\SI{25.94}{\mega\watt}
\end{equation}

\clearpage
%%%%%%%%%%%%%%%%%%%%%%%%%%%%%%%%%%%%%%%%%%%%%%%%%%%%%%%%%%%%%%%%%%%%%%%%%%%%%%%%%%%%%%%%%%%%%%%%%%%%
%%%%%%%%%%%%%%%%%%%%%%%%%%%%%%%%%%%%%%%%%%%%%%%%%%%%%%%%%%%%%%%%%%%%%%%%%%%%%%%%%%%%%%%%%%%%%%%%%%%%
\part{2011 Trafo}
\section{}
\begin{figure*}[!h]
\begin{subfigure}{.5\textwidth}
\centering
\begin{tikzpicture}[>=triangle 45,thick,node distance=0.5cm]

\path (0,0) coordinate (origin1);
\path (90:3cm) coordinate (1U);
\path (2*120+90:3cm) coordinate (1V);
\path (1*120+90:3cm) coordinate (1W);
\path (0,-6cm) coordinate (origin2);
\path (origin2) ++(90+180:3cm) coordinate (2U);
\path (origin2) ++(90+180+2*120:3cm) coordinate (2V);
\path (origin2) ++(90+180+1*120:3cm) coordinate (2W);
\path (0.7cm,1.7cm) coordinate (U1U1N);
\path (origin2) ++(0.8cm,-1.7cm) coordinate (U2U2N);

\path (240:3cm) coordinate (Usek);
\path (240:1cm) coordinate (arc1);
\path (240:2.5cm) coordinate (arc2);
\path (90:1.3cm) coordinate (arc3);
\path (-0.3cm, 1.3cm) coordinate (arc3help);

\draw [->] (origin1) -- (1U);
\draw [->] (origin1) -- (1V);
\draw [->] (origin1) -- (1W);

\draw [->] (origin2) -- (2U);
\draw [->] (origin2) -- (2V);
\draw [->] (origin2) -- (2W);

\node [above of =1U] {1U};
\node [below right of =1V] {1V};
\node [below left of =1W] {1W};
\node [below left of =2U] {2U};
\node [above right of =2V] {2V};
\node [right of =2W] {2W};
\node [right of =U1U1N] {$U_{1U1N}=U_a$};
\node [right of =U2U2N] {$U_{2U2N}=-U_d$};

\end{tikzpicture}
\end{subfigure}%
\begin{subfigure}{.5\textwidth}
\centering
\begin{circuitikz}

%\draw [help lines] (-1,-1) grid (6,-10); %Zeichnet Raster und vereinfacht damit das Zeichnen
	
	%U
	\draw (0,0) node[above=5mm] {$1U$}
	to[american inductor, o-] (0,-3);
	\draw (0,-0.7) node[right=1.5mm] {$\bullet$};

	\draw (0,-5) [american inductor, -o] 
	to node[below=10mm] {$2U$} (0,-9);
	\draw (0,-6.2) node[right=1.5mm] {$\bullet$};

	%V
	\draw (2,0) node[above=5mm] {$1V$}
	to[american inductor, o-] (2,-3);
	\draw (2,-0.7) node[right=1.5mm] {$\bullet$};

	\draw (2,-5) [american inductor, -o]
	to node[below=10mm] {$2V$} (2,-9);
	\draw (2,-6.2) node[right=1.5mm] {$\bullet$};
	
	%V
	\draw (4,0) node[above=5mm] {$1W$}
	to[american inductor, o-] (4,-3);
	\draw (4,-0.7) node[right=1.5mm] {$\bullet$};

	\draw (4,-5) [american inductor, -o] 
	to node[below=10mm] {$2W$} (4,-9);
	\draw (4,-6.2) node[right=1.5mm] {$\bullet$};
	
	%Verbindung
	\draw (0,-3) to[short,-*] (2,-3)
	-- (4,-3);
	\draw (0,-5) to[short,-*] (2,-5)
	-- (4,-5);

	%Spannungspfeile
	\draw (-0.2,0) to [open, v>=$U_a$] (-0.2,-3);
	\draw (1.8,0) to [open, v>=$U_b$] (1.8,-3);
	\draw (3.8,0) to [open, v>=$U_c$] (3.8,-3);

	\draw (-0.2,-5.5) to [open, v>=$U_d$] (-0.2,-8.5);
	\draw (1.8,-5.5) to [open, v>=$U_e$] (1.8,-8.5);
	\draw (3.8,-5.5) to [open, v>=$U_f$] (3.8,-8.5);
	
\end{circuitikz}
\end{subfigure}%
\end{figure*}

\section{}
\begin{equation}
\frac{N_1}{N_2}=\frac{\UPNS}{\USNS}=\frac{\frac{\SI{132}{\kilo\volt}}{\sqrt{3}}}{\frac{\SI{15}{\kilo\volt}}{\sqrt{3}}}=\num{8.8}
\end{equation}

\section{}
\begin{align}
\IPNS&=\frac{S_N}{3\UPNS}=\frac{\SI{80}{\mega\volt\ampere}}{3\cdot\frac{\SI{132}{\kilo\volt}}{\sqrt{3}}}=\SI{349.9}{\ampere}\\
\IPNA&=\IPNS=\SI{349.9}{\ampere}\\
\ISNS&=\frac{S_N}{3\USNS}=\frac{\SI{80}{\mega\volt\ampere}}{3\cdot\frac{\SI{15}{\kilo\volt}}{\sqrt{3}}}=\SI{3079}{\ampere}\\
\ISNA&=\ISNS=\SI{3079}{\ampere}
\end{align}

\section{}
\subsection{}
\begin{equation}
u_1=1\quad i=1\quad u_2=1
\end{equation}
\begin{align}
u_1\ce&=u_2\cz+u_R i\\
u_1\se&=u_2\sz+u_X i\\
u_1^2&=\mybr{u_1\ce}^2+\mybr{u_1\se}^2\\
&=\mybr{u_2\cz+u_R i}^2+\mybr{u_2\sz+u_X i}^2\\
&=u_R^2i^2+u_X^2i^2+u_2^2+2u_R i u_2 \cz+2u_X i u_2\sz\\
1&=u_X^2+u_2^2+2u_X\sz\\
0&=2\sz u_X+u_X^2\\
\sz&=-\frac{u_X}{2}=-\frac{\num{0.09}}{2}=\num{-0.045}\\
\cz&=\sqrt{1-\sz^2}=\sqrt{1-\num{0.045}^2}=\num{0.9990}
\end{align}
\begin{align}
\z_L&=\frac{u_2}{i}\e^{\j\varphi_2}=\cz+\j\sz=\num{0.9990}-\j\num{0.045}\\
\Z_L&=\z_L\frac{3\USNA^2}{S_N}=\mybr{\num{0.9990}-\j\num{0.045}}\frac{3\cdot\mybr{\SI{15}{\kilo\volt}}^2}{\SI{80}{\mega\volt\ampere}}=\SI{8.429}{\ohm}+\j\SI{0.3797}{\ohm}=R'+\j X'
\end{align}
\begin{align}
\frac{R\cdot\frac{1}{\j\omega C}}{R+\frac{1}{\j\omega C}}&=\frac{R}{1+\j\omega R C}=R'+\j X'\\
R&=\mybr{R'+\j\omega C'}\mybr{1+\j\omega R C}\\
\text{Realteil:}\\
R&=R'-X'\omega R C\\
\text{Imaginärteil:}\\
0&=X'+R'\omega R C\\
\omega R C &= -\frac{X'}{R'}\\
R &= R' + \frac{X'^2}{R'}=\SI{8.429}{\ohm} + \frac{\mybr{\SI{0.3797}{\ohm}}^2}{\SI{8.429}{\ohm}}=\SI{8.446}{\ohm}\\
C &= -\frac{X'}{R'\omega R}=\frac{\SI{0.3797}{\ohm}}{\SI{8.429}{\ohm}\cdot2\pi\SI{60}{\hertz}\cdot\SI{8.446}{\ohm}}=\SI{14.15}{\micro\farad}
\end{align}

\subsection{}
\begin{equation}
P_2=S_N u_2 i \cz=\SI{80}{\mega\volt\ampere}\cdot\num{1}\cdot\num{1}\cdot\num{0.9990}=\SI{79.92}{\mega\watt}
\end{equation}

\section{}
\begin{align}
I_{2K}&=\frac{\ISNA}{\left| u_R+\j u_X +\j x_{Netz}\right|}=\frac{\ISNA}{\left| u_R+\j u_X +\j X_{Netz}\frac{S_N}{3U_S^2}\right|}\\
&=\frac{\SI{3079}{\ampere}}{\left| \num{0}+\j\num{0.09}+\j\SI{4.598}{\ohm}\frac{\SI{80}{\mega\volt\ampere}}{3\cdot\mybr{\frac{\SI{132}{\kilo\volt}}{\sqrt{3}}}^2}\right|}=\SI{27.71}{\kilo\ampere}
\end{align}

\section{}
gleiche magnetische Beanspruchung:
\begin{equation}
\hat\Phi=\text{const.}
\end{equation}
aus
\begin{equation}
U=\frac{1}{\sqrt{2}}N\omega\hat\Phi
\end{equation}
folgt
\begin{align}
U_{1N,50}&=\frac{f_{N,50}}{f_{N}}U_{1N}=\frac{\SI{50}{\hertz}}{\SI{60}{\hertz}}\SI{132}{\kilo\volt}=\SI{110}{\kilo\volt}\\
U_{2N,50}&=\frac{f_{N,50}}{f_{N}}U_{2N}=\frac{\SI{50}{\hertz}}{\SI{60}{\hertz}}\SI{15}{\kilo\volt}=\SI{12.5}{\kilo\volt}
\end{align}
gleiche thermische Beanspruchung:
\begin{equation}
R_k I^2 = \text{const.}
\end{equation}
da sich R mit der Frequenz nicht ändert bleibt der Strom gleich
\begin{align}
I_{1N,50}&=I_{1N}=\SI{349.9}{\ampere}\\
I_{2N,50}&=I_{2N}=\SI{3079}{\ampere}
\end{align}
\begin{equation}
S_{N,50}=3U_{1N,50,Str}I_{1N,50}=3\cdot\frac{\SI{110}{\kilo\volt}}{\sqrt{3}}\cdot\SI{349.9}{\ampere}=\SI{66.66}{\mega\volt\ampere}
\end{equation}
\begin{align}
u_{R,50}&=R_k\frac{S_{N,50}}{3U_{1N,50,Str}^2}=u_R\frac{3\UPNS^2}{S_N}\frac{S_{N,50}}{3U_{1N,50,Str}^2}=u_R\frac{\frac{5}{6}}{\mybr{\frac{5}{6}}^2}=\num{0.0}\cdot\frac{6}{5}=\num{0.0}\\
u_{X,50}&=L_k\frac{S_{N,50}\omega_{N,50}}{3U_{1N,50,Str}^2}=u_X\frac{3\UPNS^2}{S_N\omega_N}\frac{S_{N,50}\omega_{N,50}}{3U_{1N,50,Str}^2}=\\
&=u_X\frac{\mybr{\frac{5}{6}}^2}{\mybr{\frac{5}{6}}^2}=u_X=\num{0.09}\\
u_{K,50}&=\sqrt{u_{R,50}^2+u_{X,50}^2}=0.09
\end{align}

\clearpage
%%%%%%%%%%%%%%%%%%%%%%%%%%%%%%%%%%%%%%%%%%%%%%%%%%%%%%%%%%%%%%%%%%%%%%%%%%%%%%%%%%%%%%%%%%%%%%%%%%%%
%%%%%%%%%%%%%%%%%%%%%%%%%%%%%%%%%%%%%%%%%%%%%%%%%%%%%%%%%%%%%%%%%%%%%%%%%%%%%%%%%%%%%%%%%%%%%%%%%%%%
\part{2014 Trafo}
\section{}
\begin{figure*}[!h]
\begin{subfigure}{.5\textwidth}
\centering
\begin{tikzpicture}[>=triangle 45,thick,node distance=0.5cm]

\path (0,0) coordinate (origin1);
\path (90:3cm) coordinate (1U);
\path (2*120+90:3cm) coordinate (1V);
\path (1*120+90:3cm) coordinate (1W);
\path (0,-6.5cm) coordinate (origin2);
\path (origin2) ++(90:3cm) coordinate (2U);
\path (origin2) ++(90+2*120:3cm) coordinate (2V);
\path (origin2) ++(90+1*120:3cm) coordinate (2W);
\path (1.7cm,1.7cm) coordinate (U1U1V);
\path (-1.7cm,1.7cm) coordinate (U1W1U);
\path (0cm,-1.5cm) coordinate (U1V1W);
\path (origin2) ++(60:1.73205cm) coordinate (UAdrittel);
\path (origin2) ++(1.0cm,0.5cm) coordinate (Ud);
\path (origin2) ++(1.2cm,2.2cm) coordinate (Ue);


\path (240:3cm) coordinate (Usek);
\path (240:1cm) coordinate (arc1);
\path (240:2.5cm) coordinate (arc2);
\path (90:1.3cm) coordinate (arc3);
\path (-0.3cm, 1.3cm) coordinate (arc3help);

\draw [->] (origin1) -- (1U);
\draw [->] (origin1) -- (1V);
\draw [->] (origin1) -- (1W);

\draw [->] (1V) -- (1U);
\draw [->] (1U) -- (1W);
\draw [->] (1W) -- (1V);

\draw [->] (origin2) -- (2U);
\draw [->] (origin2) -- (2V);
\draw [->] (origin2) -- (2W);

\draw [->] (origin2) -- (UAdrittel);
\draw [->] (UAdrittel) -- (2U);

\node [above of =1U] {1U};
\node [below right of =1V] {1V};
\node [below left of =1W] {1W};
\node [above of =2U] {2U};
\node [above right of =2V] {2V};
\node [below left of =2W] {2W};
\node [right of =U1U1V] {$U_{1U1V}=-U_b$};
\node [left of =U1W1U] {$U_{1W1U}=-U_a$};
\node [below of =U1V1W] {$U_{1V1W}=-U_c$};
\node [right of =Ud] {$U_d=U_g$};
\node [right of =Ue] {$-U_e=-U_h$};

\end{tikzpicture}
\end{subfigure}%
\begin{subfigure}{.5\textwidth}
\centering
\begin{circuitikz}

%\draw [help lines] (-1,-1) grid (5,17); %Zeichnet Raster und vereinfacht damit das Zeichnen
	
	%U
	\draw (0,1) node[above=5mm] {$1U$} to[short,o-] (0,0)
	to[american inductor] (0,-3);
	\draw (0,-0.8) node[right=1.5mm] {$\bullet$};

	\draw (0,-5) node[above=5mm] {$2U$}
	to[american inductor, o-] (0,-7.5);
	\draw (0,-5.7) node[right=1.5mm] {$\bullet$};
	
	\draw (0,-8.5) to[american inductor] (0,-11);
	\draw (0,-9.2) node[right=1.5mm] {$\bullet$};

	%V
	\draw (2,1) node[above=5mm] {$1V$} to[short,o-] (2,0)
	to[american inductor] (2,-3);
	\draw (2,-0.8) node[right=1.5mm] {$\bullet$};

	\draw (2,-5) node[above=5mm] {$2V$}
	to[american inductor, o-] (2,-7.5);
	\draw (2,-5.7) node[right=1.5mm] {$\bullet$};
	
	\draw (2,-8.5) to[american inductor] (2,-11);
	\draw (2,-9.2) node[right=1.5mm] {$\bullet$};
	
	%W
	\draw (4,1) node[above=5mm] {$1W$} to[short,o-] (4,0)
	to[american inductor] (4,-3);
	\draw (4,-0.8) node[right=1.5mm] {$\bullet$};

	\draw (4,-5) node[above=5mm] {$2W$}
	to[american inductor, o-] (4,-7.5);
	\draw (4,-5.7) node[right=1.5mm] {$\bullet$};
	
	\draw (4,-8.5) to[american inductor] (4,-11);
	\draw (4,-9.2) node[right=1.5mm] {$\bullet$};
	
	%N
	\draw (-1.5,-5) node[above=5mm] {$2N$} -- (-1.5,-6);
	
	%Verbindung
	%Primär
	\draw (0,0) to[short,*-] (1,0)
	-- (1,-3)
	-- (2,-3);
	\draw (2,0) to[short,*-] (3,0)
	-- (3,-3)
	-- (4,-3);
	\draw (0,-3) -- (0,-3.5)
	-- (5,-3.5)
	-- (5,0)
	to[short,-*] (4,0);
	%Sekundär
	\draw (0,-7.5) -| (1,-11)
	-- (2,-11);
	\draw (2,-7.5) -| (3,-11)
	-- (4,-11);
	\draw (4,-7.5) -| (5,-11.5)
	-| (0,-11);
	%Neutralleiter
	\draw (-1.5,-6) -- (-1.5,-8.5)
	to[short,-*] (0,-8.5)
	to[short,-*] (2,-8.5)
	-- (4,-8.5);

	%Spannungspfeile
	\draw (-0.2,0) to [open, v>=$U_a$] (-0.2,-3);
	\draw (1.8,0) to [open, v>=$U_b$] (1.8,-3);
	\draw (3.8,0) to [open, v>=$U_c$] (3.8,-3);

	\draw (-0.2,-5) to [open, v>=$U_d$] (-0.2,-7.5);
	\draw (1.8,-5) to [open, v>=$U_e$] (1.8,-7.5);
	\draw (3.8,-5) to [open, v>=$U_f$] (3.8,-7.5);
	
	\draw (-0.2,-8.5) to [open, v>=$U_g$] (-0.2,-11);
	\draw (1.8,-8.5) to [open, v>=$U_h$] (1.8,-11);
	\draw (3.8,-8.5) to [open, v>=$U_i$] (3.8,-11);
	
\end{circuitikz}
\end{subfigure}%
\end{figure*}


\section{}
\begin{align}
\frac{N_1}{\frac{N_2}{2}}&=\frac{\UPNS}{\frac{\USNA}{3}}\\
\frac{N_1}{N_2}&=\frac{3\UPNS}{2\USNA}=\frac{3\cdot\SI{12}{\kilo\volt}}{2\cdot\SI{210}{\volt}}=\num{85.71}
\end{align}

\section{}
\begin{align}
\IPNS&=\frac{S_N}{3\UPNS}=\frac{\SI{800}{\kilo\volt\ampere}}{3\cdot\SI{12}{\kilo\volt}}=\SI{22.22}{\ampere}\\
\IPNA&=\sqrt{3}\IPNS=\sqrt{3}\cdot\SI{22.22}{\ampere}=\SI{38.49}{\ampere}\\
\ISNS&=\frac{S_N}{3\USNS}=\frac{\SI{800}{\kilo\volt\ampere}}{3\frac{\SI{210}{\volt}}{\sqrt{3}}}=\SI{2199}{\ampere}\\
\ISNA&=\ISNS=\SI{2199}{\ampere}
\end{align}

\section{}
\subsection{}
\begin{align}
\tan\mybr{\varphi_2}&=\frac{X_2}{R_2}\\
L_2&=\frac{R_2\tan\mybr{\varphi_2}}{2\pi f_N}=\frac{\SI{0.16}{\ohm}\tan\mybr{\arccos\mybr{0.8}}}
{2\pi\SI{60}{\hertz}}=\SI{318.3}{\micro\henry}
\end{align}

\subsection{}
\begin{align}
\z_L&=\mybr{R_2+\j 2\pi f_N L_2}\frac{S_N}{3\USNS^2}=\mybr{\SI{0.16}{\ohm}+\j2\pi\SI{60}{\hertz}\cdot\SI{318.3}{\micro\henry}}\frac{\SI{800}{\kilo\volt\ampere}}{3\mybr{\SI{210}{\volt}}^2}\\
&=\num{0.9675}+\j\num{0.7256}\label{eq:alterWiderstand}\\
i&=\frac{u_2}{\left|\j u_X+\z\right|}=\frac{1}{\left|\j\num{0.04}+\num{0.9675}+\j\num{0.7256}\right|}=\num{0.8105}\\
\ISS&=i\ISNS=\num{0.8105}\cdot\SI{2199}{\ampere}=\SI{1782}{\ampere}\\
\ISA&=i\ISNA=\num{0.8105}\cdot\SI{2199}{\ampere}=\SI{1782}{\ampere}
\end{align}

\subsection{}
\begin{align}
u_2&=i\cdot\left|\z\right|=\num{0.8105}\cdot\left|\num{0.9675}+\j\num{0.7256}\right|=\num{0.9802}\\
\USA&=u_2\USNA=\num{0.9802}\cdot\SI{210}{\volt}=\SI{205.8}{\volt}\\
\USS&=u_2\USNS=\num{0.9802}\cdot\frac{\SI{210}{\volt}}{\sqrt{3}}=\SI{118.8}{\volt}
\end{align}

\section{}
\subsection{}
\begin{equation}
u_1=1\quad u_2=1
\end{equation}
\begin{align}
u_1\ce&=u_2\cz+u_R i\\
u_1\se&=u_2\sz+u_X i\\
\ce&=\cz
\end{align}
aus der Skizze
\begin{figure*}[!h]
\centering
\begin{tikzpicture}[>=triangle 45,thick,node distance=0.5cm]

\path (0,0) coordinate (origin1);
\path (35:3cm) coordinate (u1);
\path (35:1.5cm) coordinate (u1name);
\path (360-35:3cm) coordinate (u2);
\path (360-35:1.5cm) coordinate (u2name);
\path (3cm,0cm) coordinate (uxname);
\path (360-35:1.5cm) coordinate (u2name);
\path (0:2cm) coordinate (i);
\path (2cm,-0.1cm) coordinate (iname);
\path (35:1.5cm) coordinate (arc1);
\path (360-35:1.5cm) coordinate (arc2);
\path (1cm,-0.2cm) coordinate (arc1name);
\path (1cm,0.2cm) coordinate (arc2name);
\path (0:3cm) coordinate (arc3);

\draw [->] (origin1) -- (u1);
\draw [->] (origin1) -- (u2);
\draw [->] (u2) -- (u1);
\draw [<-] (arc1) arc (35:0:1.5cm);
\draw [<-] (arc2) arc (-35:0:1.5cm);
\draw [->] (origin1) -- (i);
\draw [-]  (arc3) arc (0:360:3cm);

\node [above of =u1name] {$u_1$};
\node [below of =u2name] {$u_2$};
\node [right of =uxname] {$iu_X$};
\node [above of =iname] {$i$};
\node [above of =arc1name] {$\varphi_1$};
\node [below of =arc2name] {$\varphi_2$};

\end{tikzpicture}
\end{figure*}

folgt
\begin{align}
\varphi_1&=-\varphi_2\\
\se&=\sz+i u_X\\
\sz&=-\frac{i u_X}{2}\label{eq:sinphi21}\\
i&=\frac{u_2}{z_L}=u_2 y_L\\
i&=y_L\\
\z_L&=\frac{\u_2}{\i}=z_L\e^{\j\varphi_2}\\
\y_L&=\frac{\i}{\u_2}=y_L\e^{-\j\varphi_2}=y_L\cz-y_L\j\sz=g_2+\j b_2\\
y_L\sz&=-b_2\\
\sz&=-\frac{b_2}{i}\label{eq:sinphi22}\\
\text{aus \eqref{eq:sinphi21} und \eqref{eq:sinphi22} folgt}\nonumber\\
-\frac{i u_X}{2}&=-\frac{b_2}{i}\\
i^2&=\frac{2 b_2}{u_X}=y^2=g_2^2+b_2^2\\
0&=b_2^2-\frac{2 b_2}{u_X}+g_2^2\\
\y_L'&=\frac{1}{\z_L'}=\frac{1}{\num{0.9675}+\j\num{0.7256}}=\num{0.6615}-\j\num{0.4961}=g_2'+\j b_2'\\
\end{align}
mit $\z_L'$ aus \eqref{eq:alterWiderstand}. 
\begin{align}
b_2&=\frac{\frac{2}{u_X}\pm\sqrt{\mybr{\frac{2}{u_X}}^2-4\cdot 1\cdot g_2^2}}{2\cdot1}\\
&=\frac{\frac{2}{\num{0.04}}\pm\sqrt{\mybr{\frac{2}{\num{0.04}}}^2-4\cdot 1\cdot \num{0.6615}^2}}{2\cdot1}\\
b_{21}&=\num{0.008753}\\
b_{22}&=\num{49.99}\quad \text{führt zu einer Überlastung des Trafos (\SI{111,0}{\kilo\ampere})}\\
b_C&=b_{21}-b_2'= \num{0.008753}+\num{0.4961} = 0.5049\\
C&=\frac{b_C}{\omega Z_{Bez}}=\frac{b_C S_N}{2\pi f_N 3 U_S^2}=\frac{0.5049\cdot\SI{800}{\kilo\volt\ampere}}{2\pi\SI{60}{\hertz}\cdot 3\mybr{\SI{210}{\volt}}^2}=\SI{8.098}{\milli\farad}
\end{align}
Hinweis: Der Lösungsweg stammt von Prof. Schmidt. Laut ihm ist das der kürzeste Rechenweg für dieses Beispiel.

\subsection{}
\begin{align}
\ISA&=i\ISNA=y_L\ISNA=\sqrt{g_2'^2+b_C^2}\ISNA\\
&=\sqrt{\num{0.6615}^2+\num{0.008753}^2}\cdot\SI{2199}{\ampere}=\SI{1455}{\ampere}
\end{align}

\clearpage
%%%%%%%%%%%%%%%%%%%%%%%%%%%%%%%%%%%%%%%%%%%%%%%%%%%%%%%%%%%%%%%%%%%%%%%%%%%%%%%%%%%%%%%%%%%%%%%%%%%%
%%%%%%%%%%%%%%%%%%%%%%%%%%%%%%%%%%%%%%%%%%%%%%%%%%%%%%%%%%%%%%%%%%%%%%%%%%%%%%%%%%%%%%%%%%%%%%%%%%%%
\part{2014 Nachtest Trafo}
\section{}
\begin{figure*}[!h]
\begin{subfigure}{.55\textwidth}
\centering
\begin{tikzpicture}[>=triangle 45,thick,node distance=0.5cm]

\path (0,0) coordinate (origin1);
\path (90:3cm) coordinate (1U);
\path (2*120+90:3cm) coordinate (1V);
\path (1*120+90:3cm) coordinate (1W);
\path (0,-6cm) coordinate (origin2);
\path (origin2) ++(90+150:3cm) coordinate (2U);
\path (origin2) ++(90+150+2*120:3cm) coordinate (2V);
\path (origin2) ++(90+150+1*120:3cm) coordinate (2W);
\path (0.7cm,1.7cm) coordinate (U1U1N);
\path (origin2) ++(-2.5cm,0cm) coordinate (U2U2V);

\path (240:3cm) coordinate (Usek);
\path (240:1cm) coordinate (arc1);
\path (240:2.5cm) coordinate (arc2);
\path (90:1.3cm) coordinate (arc3);
\path (-0.3cm, 1.3cm) coordinate (arc3help);

\draw [->] (origin1) -- (1U);
\draw [->] (origin1) -- (1V);
\draw [->] (origin1) -- (1W);

\draw [->] (origin2) -- (2U);
\draw [->] (origin2) -- (2V);
\draw [->] (origin2) -- (2W);

\draw [->] (2V) -- (2U);
\draw [->] (2U) -- (2W);
\draw [->] (2W) -- (2V);

\node [above of =1U] {1U};
\node [below right of =1V] {1V};
\node [below left of =1W] {1W};
\node [below left of =2U] {2U};
\node [above right of =2V] {2V};
\node [right of =2W] {2W};
\node [right of =U1U1N] {$U_{1U1N}=U_a$};
\node [left of =U2U2V] {$U_{2U2V}=-U_d$};

\end{tikzpicture}
\end{subfigure}%
\begin{subfigure}{.45\textwidth}
\centering
\begin{circuitikz}

%\draw [help lines] (-1,-1) grid (6,-10); %Zeichnet Raster und vereinfacht damit das Zeichnen
	
	%U
	\draw (0,0) node[above=5mm] {$1U$}
	to[american inductor, o-] (0,-3);
	\draw (0,-0.7) node[right=1.5mm] {$\bullet$};

	\draw (0,-5) [american inductor, -o] 
	to node[below=10mm] {$2U$} (0,-9);
	\draw (0,-6.2) node[right=1.5mm] {$\bullet$};

	%V
	\draw (2,0) node[above=5mm] {$1V$}
	to[american inductor, o-] (2,-3);
	\draw (2,-0.7) node[right=1.5mm] {$\bullet$};

	\draw (2,-5) [american inductor, -o]
	to node[below=10mm] {$2V$} (2,-9);
	\draw (2,-6.2) node[right=1.5mm] {$\bullet$};
	
	%V
	\draw (4,0) node[above=5mm] {$1W$}
	to[american inductor, o-] (4,-3);
	\draw (4,-0.7) node[right=1.5mm] {$\bullet$};

	\draw (4,-5) [american inductor, -o] 
	to node[below=10mm] {$2W$} (4,-9);
	\draw (4,-6.2) node[right=1.5mm] {$\bullet$};
	
	%Verbindung
	\draw (0,-3) to[short,-*] (2,-3)
	-- (4,-3);
	\draw (0,-5) -- (1,-5)
	-- (1,-8.5)
	to[short,-*] (2,-8.5);
	\draw (2,-5) -- (3,-5)
	-- (3,-8.5)
	to[short,-*] (4,-8.5);
	\draw (4,-5) -- (4,-4.5)
	-- (-1,-4.5)
	-- (-1,-8.5)
	to[short,-*] (0,-8.5);

	%Spannungspfeile
	\draw (-0.2,0) to [open, v>=$U_a$] (-0.2,-3);
	\draw (1.8,0) to [open, v>=$U_b$] (1.8,-3);
	\draw (3.8,0) to [open, v>=$U_c$] (3.8,-3);

	\draw (-0.2,-5.5) to [open, v>=$U_d$] (-0.2,-8.5);
	\draw (1.8,-5.5) to [open, v>=$U_e$] (1.8,-8.5);
	\draw (3.8,-5.5) to [open, v>=$U_f$] (3.8,-8.5);
	
\end{circuitikz}
\end{subfigure}%
\end{figure*}

\section{}
\begin{equation}
\frac{N_1}{N_2}=\frac{\UPNS}{\USNS}=\frac{\frac{\SI{400}{\kilo\volt}}{\sqrt{3}}}{\SI{21}{\kilo\volt}}=\num{11.00}
\end{equation}

\section{}
\begin{align}
\IPNS&=\frac{S_N}{3\UPNS}=\frac{\SI{450}{\mega\volt\ampere}}{3\frac{\SI{400}{\kilo\volt}}{\sqrt{3}}}=\SI{649.5}{\ampere}\\
\IPNA&=\IPS=\SI{649.5}{\ampere}\\
\ISNS&=\frac{S_N}{3\USNS}=\frac{\SI{450}{\mega\volt\ampere}}{3\cdot\SI{21}{\kilo\volt}}=\SI{12.37}{\kilo\ampere}\\
\ISNA&=\sqrt{3}\ISNS=\sqrt{3}\cdot\SI{12.37}{\kilo\ampere}=\SI{21.43}{\kilo\ampere}
\end{align}

\section{}
\subsection{}
\begin{equation}
u_1=1\quad i=1\quad \cz=0\quad \sz=-1
\end{equation}
\begin{align}
u_1\ce&=u_2\cz+u_R i\\
u_1\se&=u_2\sz+u_X i\\
\ce&=\cz\quad \rightarrow\quad \se = \pm 1\\
u_1\se&=1\cdot\mybr{-1}+1\cdot u_X\\
u_1\se&=\num{-0.875}\\
u_1&=\num{0.875}\quad\text{$u_1$ ist ein Betrag und somit $>0$}\\
\se&=\num{-1}\\
U_{20}&=u_1\USNA=\num{0.875}\cdot\SI{21}{\kilo\volt}=\SI{18.38}{\kilo\volt}
\end{align}

\subsection{}
\begin{equation}
U_{10,nom}=U_{20,nom}\ddot{u}=\SI{18.38}{\kilo\volt}\frac{\SI{400}{\kilo\volt}}{\SI{21}{\kilo\volt}}=\SI{350.1}{\kilo\volt}
\end{equation}

\subsection{}
\begin{equation}
U_{10,max}=U_{20,nom}\ddot{u}_{max}=\SI{18.38}{\kilo\volt}\frac{\SI{400}{\kilo\volt}+8\cdot\SI{5}{\kilo\volt}}{\SI{21}{\kilo\volt}}=\SI{385.1}{\kilo\volt}
\end{equation}

\section{}
\subsection{}
\begin{equation}
u_2=1\quad i=\num{19}\quad \cz=-0.8
\end{equation}
\begin{align}
\sz&=-\sqrt{1-\cz^2}=-\sqrt{1-{\num{0.8}}^2}=\num{-0.6}\\
u_1\ce&=u_2\cz+u_R i\\
u_1\se&=u_2\sz+u_X i\\
u_1^2&=\mybr{u_1\ce}^2+\mybr{u_1\se}^2\\
&=\mybr{u_2\cz+u_R i}^2+\mybr{u_2\sz+u_X i}^2\\
u_1&=\pm\sqrt{\mybr{u_2\cz+u_R i}^2+\mybr{u_2\sz+u_X i}^2}\\
u_1&=\pm\sqrt{\mybr{1\cdot\mybr{\num{-0.8}}}^2+\mybr{\num{-0.6}+\num{0.125}\cdot 1}^2}\\
u_{1,1}&=\num{0.9304}\\
u_{1,2}&=\num{-0.9304}\\
\end{align}
$u_1$ ist ein Betrag und kann somit nicht negativ sein, daher ist die Lösung $u_{1,1}$ richtig.
\begin{align}
\USA&=u_1\USNA=\num{0.9304}\cdot\SI{21}{\kilo\volt}=\SI{19.54}{\kilo\volt}
\end{align}

\subsection{}
\begin{align}
\ddot{u}&=\frac{U_{Netz}}{U_{20}}=\frac{U_{1,AL}}{\USNA}\\
&=\frac{\SI{400}{\kilo\volt}}{\SI{19.54}{\kilo\volt}}=\frac{\SI{400}{\kilo\volt}+x\cdot\SI{5}{\kilo\volt}}{\SI{21}{\kilo\volt}}\\
x&=\frac{\frac{\SI{400}{\kilo\volt}}{\SI{19.54}{\kilo\volt}}\cdot\SI{21}{\kilo\volt}-\SI{400}{\kilo\volt}}{\SI{5}{\kilo\volt}}=5.978\approx 6
\end{align}

\subsection{}
\begin{equation}
P_2=S_N u_2 i \cz=\SI{450}{\mega\volt\ampere}\cdot\num{1}\cdot\num{1}\cdot\mybr{\num{-0.8}}=\SI{-360}{\mega\watt}
\end{equation}

\clearpage
%%%%%%%%%%%%%%%%%%%%%%%%%%%%%%%%%%%%%%%%%%%%%%%%%%%%%%%%%%%%%%%%%%%%%%%%%%%%%%%%%%%%%%%%%%%%%%%%%%%%
%%%%%%%%%%%%%%%%%%%%%%%%%%%%%%%%%%%%%%%%%%%%%%%%%%%%%%%%%%%%%%%%%%%%%%%%%%%%%%%%%%%%%%%%%%%%%%%%%%%%
\part{2014 ASM}
\section{}
\subsection{}
\begin{align}
U_{Bez}&=\sqrt{2}U_{N,Str}=\sqrt{2}\frac{\SI{400}{\volt}}{\sqrt{3}}=\SI{326.6}{\volt}\\
I_{Bez}&=\sqrt{2}I_{N_Str}=\sqrt{2}\SI{25}\ampere=\SI{35.36}{\ampere}\\
\omega_{Bez}&=2\pi f_N=2 \pi \SI{75}{\hertz}=\SI{471.2}{\per\second}\\
\Psi_{Bez}&=\frac{U_{Bez}}{\omega_{Bez}}=\frac{\SI{326.6}{\volt}}{\SI{471.2}{\per\second}}=\SI{0.6931}{\volt\second}\\
t_{Bez}&=\frac{1}{\omega_{Bez}}=\frac{1}{\SI{471.2}{\per\second}}=\SI{0.002122}{\second}\\
Z_{Bez}&=\frac{U_{Bez}}{I_{Bez}}=\frac{\SI{326.6}{\volt}}{\SI{35.36}{\ampere}}=\SI{9.236}{\ohm}\\
P_{Bez}&=3U_{N,Str}I_{N,Str}=3\cdot\frac{\SI{400}{\volt}}{\sqrt{3}}\cdot\SI{25}{\ampere}=\SI{17.32}{\kilo\watt}\\
M_{Bez}&=\frac{P_{Bez}p}{\omega_{Bez}}=\frac{\SI{17.32}{\kilo\watt}\cdot 3}{\SI{471.2}{\per\second}}=\SI{110.3}{\newton\metre}
\end{align}

\subsection{}
\begin{equation}
n=\frac{f_N\cdot\SI{60}{\second\per\minute}}{p}=\frac{\SI{75}{\hertz}\cdot\SI{60}{\second\per\minute}}{3}=\SI{1850}{\per\minute}
\end{equation}

\subsection{}
\begin{align}
I_0&=\frac{U_{LL,Str}}{\left|R_{S,Str}+\j X_{S,Str,60}\right|}\\
R_{S,Str}&=\frac{U_{LL,Str}}{I_{0,\mybr{\SI{266}{\volt},\SI{50}{\hertz}}}}\cos\mybr{\varphi_0}=\frac{\frac{\SI{266}{\volt}}{\sqrt{3}}}{\SI{7.136}{\ampere}}\cdot\num{0.00857}=\SI{0.1845}{\ohm}\\
X_{S,Str,50}&=\sqrt{\mybr{\frac{U_{LL,Str}}{I_{0,\mybr{\SI{266}{\volt},\SI{50}{\hertz}}}}}^2-R_{S,Str}^2}=\sqrt{\mybr{\frac{\frac{\SI{266}{\volt}}{\sqrt{3}}}{\SI{7.136}{\ampere}}}^2-\mybr{\SI{0.1845}{\ohm}}^2}\\
&=\SI{21.52}{\ohm}\\
X_{S,Str,75}=&\frac{\SI{75}{\hertz}}{\SI{50}{\hertz}}X_{S,Str,50}=\frac{\SI{60}{\hertz}}{\SI{50}{\hertz}}\SI{21.52}{\ohm}=\SI{32.28}{\ohm}\\
X_R&=\mybr{1-\sigma}X_S=\mybr{1-\num{0.08}}\SI{32.28}=\SI{29.70}{\ohm}\\
X_\sigma&=\sigma X_S=\num{0.08}\SI{32.28}=\SI{2.582}{\ohm}\\
r_S&=\frac{R_S}{Z_{Bez}}=\frac{\SI{0.1845}{\ohm}}{\SI{9.236}{\ohm}}=\num{0.01998}\\
x_S&=\frac{X_S}{Z_{Bez}}=\frac{\SI{32.28}{\ohm}}{\SI{9.236}{\ohm}}=\num{3.495}\\
x_R&=\frac{X_R}{Z_{Bez}}=\frac{\SI{29.70}{\ohm}}{\SI{9.236}{\ohm}}=\num{3.216}\\
x_\sigma&=\frac{X_\sigma}{Z_{Bez}}=\frac{\SI{2.582}{\ohm}}{\SI{9.236}{\ohm}}=\num{0.2799}
\end{align}

\section{}
\begin{align}
\u_S&=r_S\i_S+\frac{\d\Psi_S}{\d\tau}+\j\omega_K\Psi_S\\
\u_R&=0=r_R\i_R+\frac{\d\Psi_R}{\d\tau}+\j\mybr{\omega_K-\omega_m}\Psi_R\\
\PPsi_S&=l_S\i_S+\mybr{1-\sigma}l_S\i_R\\
\PPsi_R&=l_S\mybr{1-\sigma}\mybr{\i_S+\i_R}\\
\i_R&=0\quad \text{wegen Leerlauf}\\
\PPsi_S&=l_s\i_S=l_s i_S\e^{\j\omega\tau+\varphi_i}\\
\omega_K&=0\quad\text{statorfestes Koordinatensystem}\\
\u_S&=r_S\i_S+\frac{\d}{\d\tau}\mybr{l_s i_S\e^{\j\omega\tau+\varphi_i}}\\
&=r_S\i_S+\j\omega l_S \i_S\\
\left|\i_S\right|&=\frac{\left|\u_S\right|}{\left|r_S+\j\omega l_S\right|}=\frac{1}{\sqrt{\num{0.02}^2+\mybr{\num{1.5}\cdot\num{3.5}}^2}}=\num{0.1905}\\
\left|\PPsi_S\right|&=l_S\left|\i_S\right|=\num{3.5}\cdot\num{0.1905}=\num{0.6668}\\
\left|\PPsi_R\right|&=\mybr{1-\sigma}l_S\left|\i_S\right|=\mybr{1-\num{0.08}}\num{3.5}\cdot\num{0.1905}=\num{0.6134}
\end{align}

\section{}
\subsection{}
\begin{align}
i_1&=\Re\mybr{\i_S\e^{\j\SI{0}{\degree}}}=\Re\mybr{\mybr{\num{-0.346}+\j\num{0.785}}\e^{\j\SI{0}{\degree}}}=\num{-0.346}\\
i_2&=\Re\mybr{\i_S\e^{-\j\SI{120}{\degree}}}=\Re\mybr{\mybr{\num{-0.346}+\j\num{0.785}}\e^{-\j\SI{120}{\degree}}}=\num{0.8528}\\
i_1&=\Re\mybr{\i_S\e^{-\j\SI{240}{\degree}}}=\Re\mybr{\mybr{\num{-0.346}+\j\num{0.785}}\e^{-\j\SI{240}{\degree}}}=\num{-0.5068}\\
I_1&=i_1I_{Bez}=\num{-0.346}\cdot\SI{35.36}{\ampere}=\SI{-12.23}{\ampere}\\
I_2&=i_2I_{Bez}=\num{0.8528}\cdot\SI{35.36}{\ampere}=\SI{30.16}{\ampere}\\
I_3&=i_3I_{Bez}=\num{-0.5068}\cdot\SI{35.36}{\ampere}=\SI{-17.92}{\ampere}
\end{align}

\subsection{}
\begin{align}
\i_{S,xy}&=\i_{S,\alpha\beta}\e^{-\j\gamma}=\mybr{\num{-0.346}+\j\num{0.785}}\e^{-\j\SI{45}{\degree}}=\num{0.3104}+\j\num{0.7997}\\
\PPsi_{R,xy}&=l_S\mybr{1-\sigma}\mybr{\i_{S,xy}+\i_{R,xy}}
\end{align}
Im xy-Koordinatensystem liegt in $\PPsi_{R,xy}$ in der x-Achse und ist somit rein reell. Daraus folgt, dass 
\begin{equation}
\Im\mybr{\PPsi_{R,xy}}=0\quad \text{bzw.}\quad \Im\mybr{\i_{S,xy}}=-\Im\mybr{\i_{R,xy}}.
\end{equation}
Aus
\begin{equation}
\u_R=0=r_R\i_{R,xy}+0+\j\mybr{\omega_K-\omega_m}\PPsi_{R,xy}\\
\end{equation}
folgt
\begin{equation}
\Re\mybr{\i_{R,xy}}=0
\end{equation}
und damit
\begin{equation}
\i_{R,xy}=-\j\Im\mybr{\i_{S,xy}}=-\j\num{0.7997}.
\end{equation}
\begin{align}
\i_{R,\alpha\beta}&=\i_{R,xy}\e^{\j\gamma}=-\j\num{0.7997}\cdot\e^{\j\SI{45}{\degree}}=\SI{0.5655}-\j\num{0.5655}\\
\PPsi_{R,\alpha\beta}&=\mybr{1-\sigma}l_S\mybr{\i_{S,\alpha\beta}+\i_{R,\alpha\beta}}\\
&=\mybr{1-\num{0.08}}\num{3.5}\mybr{\num{-0.346}+\j\num{0.785}+\SI{0.5655}-\j\num{0.5655}}=\num{0.7068}+\j\num{0.7068}\\
\PPsi_{S,\alpha\beta}&=\PPsi_{R,\alpha\beta}+\sigma l_S\i_{S,\alpha\beta}\\
&=\num{0.7068}+\j\num{0.7068}+\num{0.08}\cdot\num{3.5}\mybr{\num{-0.346}+\j\num{0.785}}=\num{0.6098}+\j\num{0.9265}
\end{align}

\subsection{}
\begin{equation}
m=-\Im\mybr{\i_{S,\alpha\beta}^*\PPsi_{S,\alpha\beta}}=-\Im\mybr{\mybr{\num{-0.346}+\j\num{0.785}}\mybr{\num{0.6098}+\j\num{0.9265}}}=\num{0.7993}
\end{equation}

\section{}
\subsection{}
\begin{equation}
\PPsi_{S,\alpha\beta}=\PPsi_{R,\alpha\beta}+\sigma l_S\i_{S,\alpha\beta}=\num{0.5}+\j\num{0.866}+\num{0.1}\cdot\num{3.5}\mybr{\num{-0.274}+\j\num{0.525}}=\num{0.4041}+\j\num{1.050}
\end{equation}
\begin{figure*}[!hp]
\centering
\begin{tikzpicture}[>=triangle 45,thick,node distance=0.5cm]

\path (0,0) coordinate (origin1);
\path (1*8,0) coordinate (alpha);
\path (0,1*8) coordinate (beta);
\path (-0.274*6,0.525*6) coordinate (is);
\path (0.5*6,0.866*6) coordinate (PsiR);
\path (0.4041*6,1.050*6) coordinate (PsiS);
\path (-0.4111*6,0.1582*6) coordinate (m1);
\path (m1) ++(0.4041*6,1.050*6) coordinate (m2);

\fill[gray!30,nearly transparent] (origin1) -- (m1) -- (m2) -- (PsiS) -- cycle;

\draw [->] (origin1) -- (alpha);
\draw [->] (origin1) -- (beta);
\draw [->] (origin1) -- node[below left]{$\i_S$} (is);
\draw [->] (origin1) -- node[below right]{$\PPsi_R$} (PsiR);
\draw [->] (origin1) -- node[above left]{$\PPsi_S$} (PsiS);
\draw [->] (PsiR) -- node[above right]{$\PPsi_\sigma$} (PsiS);
\draw [-] (origin1) -- (m1);
\draw [-] (m1) -- (m2);
\draw [-] (PsiS) -- node[below=0.7cm] {$m$} (m2);

\node [below of =alpha] {$\alpha$};
\node [right of =beta] {$\beta$};

\end{tikzpicture}
\end{figure*}

\subsection{}
\begin{equation}
m=-\Im\mybr{\i_{S,\alpha\beta}^*\PPsi_{S,\alpha\beta}}=-\Im\mybr{\mybr{\num{-0.274}-\j\num{0.525}}\mybr{\num{0.4041}+\j\num{1.050}}}=\num{0.4999}
\end{equation}

\clearpage
%%%%%%%%%%%%%%%%%%%%%%%%%%%%%%%%%%%%%%%%%%%%%%%%%%%%%%%%%%%%%%%%%%%%%%%%%%%%%%%%%%%%%%%%%%%%%%%%%%%%
%%%%%%%%%%%%%%%%%%%%%%%%%%%%%%%%%%%%%%%%%%%%%%%%%%%%%%%%%%%%%%%%%%%%%%%%%%%%%%%%%%%%%%%%%%%%%%%%%%%%
\part{2007 B GSM}
\section{}
\begin{figure*}[!h]
\centering
\begin{circuitikz}
\begin{scope}[scale=0.8]
	%\draw [help lines] (-1,-1) grid (7,7); %Zeichnet Raster und vereinfacht damit das Zeichnen
	
	%Ankerkreis 1
	\draw (0,7) -- (2,7)
	to [short, i>=$I_A$] (2,6);
	
	%Motor
	\draw[fill=black] (1.65,2.85) rectangle (2.35,6.15);
	\draw[fill=white] (2,4.5) ellipse (1.25 and 1.25);
	\draw [->] (3,4.5) -- (1,4.5) node [below right]{$\Phi$};
	\draw [->] (2,5.5) -- (2,3.5) node [above right]{$I_A$};
	\draw [->] (0.5,4.5) arc (180:120:1.5) node [above]{$n$};
	\draw [->] (0,4.5) arc (180:130:2) node [above]{$M$};
	
	%Ankerkreis 2
	\draw (2,3) |- (3,2.5)
	|- (2,-0.5) 
	-- (2,0) node[right] {B1}
	to [L] (2,2) node[right] {B2}
	-| (1,0)
	-- (0,0);
	
	%Erregerkreis
	\draw (6,6) to [short, i>=$I_E$] (6,4.5) node [below] {F1}
	to [L] (4,4.5) node [below] {F2}
	-- (4,6);
	
	%Spannungspfeil
	{
	\ctikzset{voltage/distance from node=.2}
	\draw (-0.5,0) to [open, v^=$U_A$] (-0.5,7);
	}
	
\end{scope}
\end{circuitikz}
\end{figure*}

\section{}
\begin{align}
P_N&=U_{A,N} I_{A,N} \eta_N\\
I_N&=\frac{P_N}{U_{A,N}\eta_N}=\frac{\SI{1}{\kilo\watt}}{\SI{220}{\volt}\cdot\num{0.85}}=\SI{5.348}{\ampere}\\
S_N&=U_{A,N}I_{A,N}=P_N+P_V=P_N+I_{A,N}^2 R_A\\
R_A&=\frac{U_{A,N}I_{A,N}-P_N}{I_{A,N}^2}=\frac{\SI{220}{\volt}\cdot\SI{5.348}{\ampere}-\SI{1}{\kilo\watt}}{\mybr{\SI{5.348}{\ampere}}^2}=\SI{6.173}{\ohm}\\
U_{A,N}&=k_1\phi n_0\\
k_1\phi&=\frac{U_{A,N}}{n_0}=\frac{\SI{220}{\volt}}{\frac{\SI{4000}{\rounds\per\minute}}{\SI{60}{\second\per\minute}}}=\SI{3.300}{\volt\second}\\
M_N&=\frac{k_1\phi}{2\pi}I_{A,N}=\frac{\SI{3.300}{\volt\second}}{2\pi}\SI{5.348}{\ampere}=\SI{2.809}{\newton\metre}\\
P_N&=M_N\omega_N\\
n_N&=\frac{P_N}{M_N 2\pi}=\frac{\SI{1}{\kilo\watt}}{\SI{2.809}{\newton\metre}\cdot 2 \pi}\SI{60}{\second\per\minute}=\SI{3400}{\rounds\per\minute}
\end{align}

\section{}
\begin{align}
\num{1.5}M_N&=\frac{k_1\phi}{2\pi}I_A\\
I_A&=\frac{2\pi\num{1.5}M_N}{k_1\phi}\\
n&=0\\
U_A&=R_A I_A=R_A \frac{2\pi\num{1.5}M_N}{k_1\phi}=\SI{6.173}{\ohm}\frac{2\pi\num{1.5}\cdot\SI{2.809}{\newton\metre}}{\SI{3.300}{\volt\second}}=\SI{49.53}{\volt}
\end{align}

\section{}
\begin{align}
U_{di\alpha}&=U_{di0}\cos\mybr{\alpha}\\
\alpha&=\arccos\mybr{\frac{U_{di\alpha}}{U_{di0}}}=\arccos\mybr{\frac{U_N}{\frac{3}{\pi}\sqrt{2}\frac{U_{1,verk}}{\ddot{u}}}}=\arccos\mybr{\frac{\SI{220}{\volt}}{\frac{3}{\pi}\sqrt{2}\frac{\SI{400}{\volt}}{\num{2.2}}}}=\SI{26.37}{\degree}
\end{align}

\section{}
\begin{align}
\num{0.6}M_N&=\frac{k_1\phi}{2\pi}I_A\\
I_A&=\frac{2\pi\num{0.6}M_N}{k_1\phi}\\
U_A&=R_A I_A + k_1\phi n_N\\
n_N&=\frac{U_A-R_A I_A}{k_1\phi}=\frac{\frac{3}{\pi}\sqrt{2}\frac{U_{1,verk}}{\ddot{u}}-R_A\frac{2\pi\num{0.6}M_N}{k_1\phi}}{k_1\phi}\\
&=\frac{\frac{3}{\pi}\sqrt{2}\frac{\SI{300}{\volt}}{\num{2.2}}-\SI{6.173}{\ohm}\frac{2\pi\num{0.6}\cdot\SI{2.809}{\newton\metre}}{\SI{3.300}{\volt\second}}}{\SI{3.300}{\volt\second}}\SI{60}{\second\per\minute}=\SI{2988}{\rounds\per\minute}
\end{align}

\section{}
\begin{align}
U_A&=\frac{3}{\pi}\sqrt{2}\frac{U_{1,verk}}{\ddot{u}}\cos\mybr{\alpha}=\frac{3}{\pi}\sqrt{2}\frac{\SI{400}{\volt}}{\num{2.2}}\cos\mybr{\SI{120}{\degree}}=\SI{-122.8}{\volt}\\
M_N&\rightarrow I_{A,N}\\
n&=\frac{U_A-R_A I_A}{k_1\phi}=\frac{U_A-R_A I_{A,N}}{k_1\phi}=\frac{\SI{-122.8}{\volt}-\SI{6.173}{\ohm}\cdot\SI{5.348}{\ampere}}{\SI{3.300}{\volt\second}}\SI{60}{\second\per\minute}\\
&=\SI{-2832}{\rounds\per\minute}
\end{align}

\clearpage
%%%%%%%%%%%%%%%%%%%%%%%%%%%%%%%%%%%%%%%%%%%%%%%%%%%%%%%%%%%%%%%%%%%%%%%%%%%%%%%%%%%%%%%%%%%%%%%%%%%%
%%%%%%%%%%%%%%%%%%%%%%%%%%%%%%%%%%%%%%%%%%%%%%%%%%%%%%%%%%%%%%%%%%%%%%%%%%%%%%%%%%%%%%%%%%%%%%%%%%%%
\part{2010 A GSM}
\section{}
\begin{figure*}[!h]
\centering
\begin{circuitikz}
\begin{scope}[scale=0.8]
	%\draw [help lines] (-1,-1) grid (7,7); %Zeichnet Raster und vereinfacht damit das Zeichnen
	
	%Ankerkreis 1
	\draw (0,7) -- (2,7)
	to [short, i>=$I_A$] (2,6);
	
	%Motor
	\draw[fill=black] (1.65,2.85) rectangle (2.35,6.15);
	\draw[fill=white] (2,4.5) ellipse (1.25 and 1.25);
	\draw [->] (3,4.5) -- (1,4.5) node [below right]{$\Phi$};
	\draw [->] (2,5.5) -- (2,3.5) node [above right]{$I_A$};
	\draw [->] (0.5,4.5) arc (180:120:1.5) node [above]{$n$};
	\draw [->] (0,4.5) arc (180:130:2) node [above]{$M$};
	
	%Ankerkreis 2
	\draw (2,3) |- (3,2.5)
	|- (2,-0.5) 
	-- (2,0) node[right] {B1}
	to [L] (2,2) node[right] {B2}
	-| (1,0)
	-- (0,0);
	
	%Erregerkreis
	\draw (6,6) to [short, i>=$I_E$] (6,4.5) node [below] {F1}
	to [L] (4,4.5) node [below] {F2}
	-- (4,6);
	
	%Spannungspfeil
	{
	\ctikzset{voltage/distance from node=.2}
	\draw (-0.5,0) to [open, v^=$U_A$] (-0.5,7);
	}
	\draw (4,6) to [open, v^=$U_E$] (6,6);
	
\end{scope}
\end{circuitikz}
\end{figure*}

\section{}
\begin{align}
I_A&=0\\
U_{A,N}&=k_1\phi\mybr{0}n_0\\
n_0&=\frac{U_{A,N}}{k_1\phi\mybr{0}}=\frac{\SI{440}{\volt}}{\SI{18}{\volt\second}}\SI{60}{\second\per\minute}=\SI{1467}{\rounds\per\minute}
\end{align}

\section{}
\begin{align}
P_V&=I_{A,N}^2 R_A=\mybr{1-\eta_N}S_N\\
R_A&=\frac{\mybr{1-\eta_N}U_{A,N} I_{A,N}}{I_{A,N}^2}=\frac{\mybr{1-\num{0.94}}\SI{440}{\volt}}{\SI{845}{\ampere}}=\SI{0.03124}{\ohm}
\end{align}

\section{}
\subsection{}
\begin{align}
M_N&=\frac{k_1\phi}{2\pi}I_{A,N}=\frac{\SI{18}{\volt\second}}{2\pi}\SI{845}{\ampere}=\SI{2421}{\newton\metre}\\
U_{A,N}&=R_A I_{A,N} + k_1\phi n_N\\
n_N&=\frac{U_{A,N}-R_A I_{A,N}}{k_1\phi}=\frac{\SI{440}{\volt}-\SI{0.03124}{\ohm}\cdot\SI{845}{\ampere}}{\SI{18}{\volt\second}}\SI{60}{\second\per\minute}=\SI{1379}{\rounds\per\minute}
\end{align}
\subsection{}
\begin{align}
M_N&=\frac{k_1\phi\mybr{I_{A,N}}}{2\pi}I_{A,N}=\frac{k_1\phi\mybr{0}\mybr{1-\num{0.07}\mybr{\frac{I_{A,N}}{I_{A,N}}}^2}}{2\pi}I_{A,N}\\
&=\frac{\SI{18}{\volt\second}\mybr{1-\num{0.07}}}{2\pi}\SI{845}{\ampere}=\SI{2251}{\newton\metre}\\
n_N&=\frac{U_{A,N}-R_A I_{A,N}}{k_1\phi\mybr{I_{A,N}}}=\frac{U_{A,N}-R_A I_{A,N}}{k_1\phi\mybr{0}\mybr{1-\num{0.07}\mybr{\frac{I_{A,N}}{I_{A,N}}}^2}}\\
&=\frac{\SI{440}{\volt}-\SI{0.03124}{\ohm}\cdot\SI{845}{\ampere}}{\SI{18}{\volt\second}\mybr{1-\num{0.07}}}\SI{60}{\second\per\minute}=\SI{1482}{\rounds\per\minute}
\end{align}

\section{}
\subsection{}
\begin{equation}
U_A=-R_A I_A + k_1\phi n=-\SI{0.03124}{\ohm}\cdot I_A+\SI{18}{\volt\second}\frac{\SI{1500}{\rounds\per\minute}}{\SI{60}{\second\per\minute}}
\end{equation}
\subsection{}
\begin{align}
U_A&=-R_A I_A + k_1\phi\mybr{I_A} n=-R_A I_A + k_1\phi\mybr{0} \mybr{1-\num{0.07}\mybr{\frac{I_A}{I_{A,N}}}^2}n\\
&=-\SI{0.03124}{\ohm}\cdot I_A+\SI{18}{\volt\second}\mybr{1-\num{0.07}\mybr{\frac{I_A}{\SI{845}{\ampere}}}^2}\frac{\SI{1500}{\rounds\per\minute}}{\SI{60}{\second\per\minute}}\\
\end{align}
\begin{figure*}[!hp]
\centering
\begin{tikzpicture}
	\begin{axis}[
        xlabel=$I_A$ in \si{\ampere},
        ylabel=$U_A$ in \si{\volt},
		grid,
		ymin=0,
		xmin=0,
		ymax=500,
		xmax=1.5*845,
		legend to name=ua-legende,
		name=uaplot]
	\addplot[domain=0:1.5*845, blue] {-0.03124*x+18*1500/60};
	\addlegendentry{$U_A$ ohne ARW}
	\addplot[domain=0:1.5*845, red] {-0.03124*x+18*(1-0.07*(x/845)^2)*1500/60};
	\addlegendentry{$U_A$ mit ARW}
    \end{axis}
	\node at (uaplot.east) [anchor=south, xshift=-2cm, yshift=-1.5cm]
			{\pgfplotslegendfromname{ua-legende}};
\end{tikzpicture}
\end{figure*}

\clearpage
%%%%%%%%%%%%%%%%%%%%%%%%%%%%%%%%%%%%%%%%%%%%%%%%%%%%%%%%%%%%%%%%%%%%%%%%%%%%%%%%%%%%%%%%%%%%%%%%%%%%
%%%%%%%%%%%%%%%%%%%%%%%%%%%%%%%%%%%%%%%%%%%%%%%%%%%%%%%%%%%%%%%%%%%%%%%%%%%%%%%%%%%%%%%%%%%%%%%%%%%%
\part{2011 A GSM}
\section{}
\begin{figure*}[!h]
\centering
\begin{circuitikz}
\begin{scope}[scale=0.8]
	%\draw [help lines] (-1,-1) grid (7,7); %Zeichnet Raster und vereinfacht damit das Zeichnen
	
	%Ankerkreis 1
	\draw (0,7) -- (2,7)
	to [short, i>=$I_A$] (2,6);
	
	%Motor
	\draw[fill=black] (1.65,2.85) rectangle (2.35,6.15);
	\draw[fill=white] (2,4.5) ellipse (1.25 and 1.25);
	\draw [->] (3,4.5) -- (1,4.5) node [below right]{$\Phi$};
	\draw [->] (2,5.5) -- (2,3.5) node [above right]{$I_A$};
	\draw [->] (0.5,4.5) arc (180:120:1.5) node [above]{$n$};
	\draw [->] (0,4.5) arc (180:130:2) node [above]{$M$};
	
	%Ankerkreis 2
	\draw (2,3) |- (3,2.5)
	|- (2,-0.5) 
	-- (2,0) node[right] {B1}
	to [L] (2,2) node[right] {B2}
	-| (1,-3.5)
	-- (2,-3.5) node[right] {C1}
	to [L] (2,-1.5) node[ right] {C2}
	|- (0,-1);
	
	%B6 links
	\draw (-7,4) to [short, o-] (-5,4)
	to [thyristor] (-5,7)
	to [short, -*] (-3.5,7);
	\draw (-7,3) to [short, o-] (-3.5,3)
	-- (-3.5,4)
	to [thyristor] (-3.5,7)
	to [short, -*] (-2,7);
	\draw (-7,2) to [short, o-] (-2,2)
	-- (-2,4)
	to [thyristor] (-2,7)
	-- (0,7);
	\draw (-3.5,-1) to [short,*-] (-5,-1)
	to [thyristor] (-5,2)
	to [short, -*] (-5,4);
	\draw (-2,-1) to [short,*-] (-3.5,-1)
	to [thyristor] (-3.5,2)
	to [short, -*] (-3.5,3);
	\draw (0,-1) -- (-2,-1)
	to [thyristor,-*] (-2,2);
	
	%B6 rechts
	\draw (12,2) to [short, o-] (10,2)
	to [thyristor] (10,-1)
	to [short, -*] (8.5,-1);
	\draw (12,3) to [short, o-] (8.5,3)
	-- (8.5,2)
	to [thyristor] (8.5,-1)
	to [short, -*] (7,-1);
	\draw (12,4) to [short, o-] (7,4)
	-- (7,2)
	to [thyristor] (7,-1)
	to [short, -*] (2,-1);
	\draw (2,7) to [short,*-] (7,7)
	to [thyristor] (7,4)
	to [short, -*] (7,4);
	\draw (7,7) to [short,*-] (8.5,7)
	to [thyristor] (8.5,4)
	to [short, -*] (8.5,3);
	\draw (8.5,7) to [short,*-] (10,7)
	to [thyristor] (10,4)
	to [short,-*] (10,2);
	
	%Erregerkreis
	\draw (6,6) to [short, i_>=$I_E$] (6,4.5) node [below] {F1}
	to [L] (4,4.5) node [below] {F2}
	-- (4,6);
	
	%Spannungspfeil
	{
	\ctikzset{voltage/distance from node=.2}
	\draw (-0.5,-1) to [open, v^=$U_A$] (-0.5,7);
	}
	\draw (4,6) to [open, v^=$U_E$] (6,6);
	
\end{scope}
\end{circuitikz}
\end{figure*}
Die erste Brücke ist in Betrieb, da es sich um Motorbetrieb und Rechtslauf handelt. Die Maschine wird fremderregt betrieben, damit auch bei kleineren Ankerspannungen $U_A$ der Fluss konstant gehalten werden kann. Außerdem gehören die Anschlussbezeichnungen F1 und F2 zu einer fremderregt Gleichstrommaschine.

\section{}
\begin{align}
P_V&=I_{A,N}^2 R_A=\num{0.8}\mybr{S_N-P_N}\\
R_A&=\frac{\num{0.8}\mybr{U_{A,N}I_{A,N}-P_N}}{I_{A,N}^2}=\frac{\num{0.8}\mybr{\SI{400}{\volt}\cdot\SI{156}{\ampere}-\SI{55}{\kilo\watt}}}{\mybr{\SI{156}{\ampere}}^2}=\SI{0.2433}{\ohm}
\end{align}

\section{}
\begin{align}
P_N&=M_N\omega_N\\
M_N&=\frac{P_N}{\omega_N}=\frac{P_N}{2\pi n_N}=\frac{\SI{55}{\kilo\watt}}{2\pi\frac{\SI{1500}{\rounds\per\minute}}{\SI{60}{\second\per\minute}}}=\SI{350.1}{\newton\metre}\\
M_N&=\frac{k_1\phi}{2\pi}I_{A,N}\\
k_1\phi&=\frac{2\pi M_N}{I_{A,N}}=\frac{2\pi\SI{350.1}{\newton\metre}}{\SI{156}{\ampere}}=\SI{14.10}{\volt\second}\\
U_{A,N}&=k_1\phi n_0\\
n_0&=\frac{U_{A,N}}{k_1\phi}=\frac{\SI{400}{\volt}}{\SI{14.10}{\volt\second}}\SI{60}{\second\per\minute}=\SI{1702}{\rounds\per\minute}
\end{align}

\section{}
\begin{align}
U_{di\alpha}&=U_{di0}\cos\mybr{\alpha}\\
\alpha&=\arccos\mybr{\frac{U_{di\alpha}}{U_{di0}}}=\arccos\mybr{\frac{U_{A,N}}{\frac{3}{\pi}\sqrt{2}U_{v,eff}}}=\arccos\mybr{\frac{\SI{400}{\volt}}{\frac{3}{\pi}\sqrt{2}\cdot\SI{400}{\volt}}}=\SI{42.23}{\degree}
\end{align}

\section{}
\subsection{}
Um konsistente Gleichungen zu erhalten, wird ein Widerstand $R_A'$ berechnet, welcher die gesamte Verlustleistung aufnimmt
\begin{equation}
R_A'=\frac{\mybr{U_{A,N}I_{A,N}-P_N}}{I_{A,N}^2}=\frac{\mybr{\SI{400}{\volt}\cdot\SI{156}{\ampere}-\SI{55}{\kilo\watt}}}{\mybr{\SI{156}{\ampere}}^2}=\SI{0.3041}{\ohm}.
\end{equation}
Für Rechtslauf gilt
\begin{equation}
U_A=R_A' I_{A,N}+k_1\phi n=\SI{0.3041}{\ohm}\cdot\SI{156}{\ampere}+\SI{14.10}{\volt\second}\frac{n}{\SI{60}{\second\per\minute}},
\end{equation}
bei Linkslauf dreht sich das Vorzeichen von $I_A$ um
\begin{equation}
U_A=R_A' I_{A,N}+k_1\phi n=\SI{0.3041}{\ohm}\cdot\mybr{\SI{-156}{\ampere}}+\SI{14.10}{\volt\second}\frac{n}{\SI{60}{\second\per\minute}}.
\end{equation}
Die maximal zulässige Spannung ist $\num{1.2}\cdot\SI{400}{\volt}=\SI{480}{\volt}$.
Die Drehzahlen ab denen der Feldschwächbetrieb verwendet werden muss berechnen sich zu
\begin{align}
n_{Feldschw\ddot{a}ch,1}&=\frac{U_{A,max}-R_A I_{A,N}}{k_1\phi}=\frac{\num{1.2}\cdot\SI{400}{\volt}-\SI{0.3041}{\ohm}\cdot\SI{156}{\ampere}}{\SI{14.10}{\volt\second}}\SI{60}{\second\per\minute}\\
&=\SI{1841}{\rounds\per\minute}\\
n_{Feldschw\ddot{a}ch,2}&=\frac{-U_{A,max}-R_A I_{A,N}}{k_1\phi}=\frac{\num{1.2}\cdot\mybr{\SI{-400}{\volt}}-\SI{0.3041}{\ohm}\cdot\mybr{\SI{-156}{\ampere}}}{\SI{14.10}{\volt\second}}\SI{60}{\second\per\minute}\\
&=\SI{-1841}{\rounds\per\minute}.
\end{align}
\begin{figure*}[!h]
\centering
\begin{tikzpicture}
	\begin{axis}[
        xlabel=$n$ in \si{\rounds\per\minute},
        ylabel=$U_A$ in \si{\volt},
		grid,
		ymin=-500,
		xmin=-3000,
		ymax=500,
		xmax=3000]
	\addplot[domain=-3000:-1840.683, blue] {-1.2*400};
	\addplot[domain=-1840.683:0, blue] {-0.3041*156+14.10/60*x};
	\addplot[domain=0:1840.683, blue] {0.3041*156+14.10/60*x};
	\addplot[domain=1840.683:3000, blue] {1.2*400};
    \end{axis}
\end{tikzpicture}
\end{figure*}

\subsection{}
Für Rechtslauf gilt
\begin{equation}
M_{i,N}=\frac{k_1\phi}{2\pi}I_{A,N}=\frac{\SI{14.10}{\volt\second}}{2\pi}\SI{156}{\ampere}=\SI{350.1}{\newton\metre},
\end{equation}
bei Linkslauf dreht sich das Vorzeichen von $I_A$ um
\begin{equation}
M_{i,N}=\frac{k_1\phi}{2\pi}I_{A,N}=\frac{\SI{14.10}{\volt\second}}{2\pi}\mybr{\SI{-156}{\ampere}}=\SI{-350.1}{\newton\metre}.
\end{equation}
Im Feldschwächbetrieb gilt
\begin{align}
M_i&=M_{i,N}\frac{n_{Felschw\ddot{a}ch,1}}{n}\\
&=\SI{350.1}{\newton\metre}\frac{\SI{1841}{\rounds\per\minute}}{n}.
\end{align}
Bei Linkslauf dreht sich das Vorzeichen von $I_A$ um
\begin{align}
M_i&=M_{i,N}\frac{n_{Felschw\ddot{a}ch,2}}{n}\\
&=\SI{-350.1}{\newton\metre}\frac{\SI{-1841}{\rounds\per\minute}}{n}.
\end{align}
\begin{figure*}[!ht]
\centering
\begin{tikzpicture}
	\begin{axis}[
        xlabel=$n$ in \si{\rounds\per\minute},
        ylabel=$M_i$ in \si{\newton\metre},
		grid,
		ymin=-400,
		xmin=-3000,
		ymax=400,
		xmax=3000]
	\addplot[domain=-3000:-1840.683, blue] {350.1*1841/x};
	\addplot[domain=-1840.683:0, blue] {-350.1};
	\addplot[domain=0:1840.683, blue] {350.1};
	\addplot[domain=1840.683:3000, blue] {350.1*1841/x};
    \end{axis}
\end{tikzpicture}
\end{figure*}

\subsection{}
Für Rechtslauf gilt
\begin{equation}
P=M_i\omega=M_i 2\pi n=\SI{350.1}{\newton\metre}\cdot 2\pi \frac{n}{\SI{60}{\second\per\minute}},
\end{equation}
bei Linkslauf gilt hingegen
\begin{equation}
P=M_i\omega=M_i 2\pi n=\SI{-350.1}{\newton\metre}\cdot 2\pi \frac{n}{\SI{60}{\second\per\minute}}
\end{equation}
Im Feldschwächbereich gilt
\begin{align}
P&=M_i\omega=M_i 2\pi n=M_{i,N}\frac{n_{Felschw\ddot{a}ch,1}}{n}2\pi n\\
&=\SI{350.1}{\newton\metre}\frac{\SI{1841}{\rounds\per\minute}}{\SI{60}{\second\per\minute}}2\pi=\SI{67.50}{\kilo\watt}
\end{align}
\begin{figure*}[!ht]
\centering
\begin{tikzpicture}
	\begin{axis}[
        xlabel=$n$ in \si{\rounds\per\minute},
        ylabel=$P$ in \si{\watt},
		grid,
		ymin=0,
		xmin=-3000,
		ymax=80000,
		xmax=3000]
	\addplot[domain=-3000:-1840.683, blue] {67500};
	\addplot[domain=-1840.683:0, blue] {-350.1*2*pi*x/60};
	\addplot[domain=0:1840.683, blue] {350.1*2*pi*x/60};
	\addplot[domain=1840.683:3000, blue] {67500};
    \end{axis}
\end{tikzpicture}
\end{figure*}

\subsection{}
\begin{align}
U_{di\alpha}&=U_{di0}\cos\mybr{\alpha}\\
\alpha&=\arccos\mybr{\frac{U_{di\alpha}}{U_{di0}}}=\arccos\mybr{\frac{U_A\mybr{n}}{\frac{3}{\pi}\sqrt{2}U_{v,eff}}}=\arccos\mybr{\frac{U_A\mybr{n}}{\frac{3}{\pi}\sqrt{2}\cdot\SI{400}{\volt}}}
\end{align}
Bei Rechtslauf ist die linke Brücke aktiv, die rechte wird nicht angesteuert
\begin{equation}
\alpha_1=\arccos\mybr{\frac{R_A' I_{A,N}+k_1\phi n}{\frac{3}{\pi}\sqrt{2}U_{v,eff}}}=\arccos\mybr{\frac{\SI{0.3041}{\ohm}\cdot\SI{156}{\ampere}+\SI{14.10}{\volt\second}\frac{n}{\SI{60}{\second\per\minute}}}{\frac{3}{\pi}\sqrt{2}\cdot\SI{400}{\volt}}},
\end{equation}
bei Linkslauf ist die rechte Brücke aktiv, die linke wird nicht angesteuert
\begin{equation}
\alpha_2=\arccos\mybr{\frac{R_A' I_{A,N}+k_1\phi n}{\frac{3}{\pi}\sqrt{2}U_{v,eff}}}=\arccos\mybr{\frac{\SI{0.3041}{\ohm}\cdot\mybr{\SI{-156}{\ampere}}+\SI{14.10}{\volt\second}\frac{n}{\SI{60}{\second\per\minute}}}{\frac{3}{\pi}\sqrt{2}\cdot\mybr{\SI{-400}{\volt}}}}.
\end{equation}
Würden beide Brücken gleichzeitig angesteuert werden, würde es zu einem Kurzschluss kommen.
\begin{figure*}[!ht]
\centering
\begin{tikzpicture}
	\begin{axis}[
        xlabel=$n$ in \si{\rounds\per\minute},
        ylabel=$\alpha$ in \si{\degree},
		grid,
		ymin=0,
		xmin=-3000,
		ymax=90,
		xmax=3000,
		legend to name=alpha-legende,
		name=alphaplot]
	
	\addplot[domain=0:1840.683, blue] {acos((0.3041*156+14.10/60*x)/(3/pi*sqrt(2)*400)};
	\addlegendentry{$\alpha_1$}
	\addplot[domain=-3000:-1840.683, red] {acos(-1.2*400/(3/pi*sqrt(2)*(-400))};
	\addlegendentry{$\alpha_2$}
	\addplot[domain=1840.683:3000, blue] {acos(1.2*400/(3/pi*sqrt(2)*400)};
	\addplot[domain=-1840.683:0, red] {acos((-0.3041*156+14.10/60*x)/(3/pi*sqrt(2)*(-400))};

    \end{axis}
	\node at (alphaplot.east) [anchor=south, xshift=-1.50cm, yshift=1.5cm]
			{\pgfplotslegendfromname{alpha-legende}};
\end{tikzpicture}
\end{figure*}

\subsection{}
Außerhalb des Intervalls $\mybr{\SI{-1841}{\rounds\per\minute},\SI{1841}{\rounds\per\minute}}$ ist ein Feldschwächbetrieb erforderlich.

\clearpage
%%%%%%%%%%%%%%%%%%%%%%%%%%%%%%%%%%%%%%%%%%%%%%%%%%%%%%%%%%%%%%%%%%%%%%%%%%%%%%%%%%%%%%%%%%%%%%%%%%%%
%%%%%%%%%%%%%%%%%%%%%%%%%%%%%%%%%%%%%%%%%%%%%%%%%%%%%%%%%%%%%%%%%%%%%%%%%%%%%%%%%%%%%%%%%%%%%%%%%%%%
\part{2013 GSM}
\section{}
\subsection{}
\begin{align}
P_N&=\eta_N U_{A,N} I_{A,N}\\
I_{A,N}&=\frac{P_N}{\eta_N U_{A,N}}=\frac{\SI{22}{\kilo\watt}}{\num{0.78}\cdot\SI{440}{\volt}}=\SI{64.10}{\ampere}
\end{align}

\subsection{}
\begin{align}
P_V&=I_{A,N}^2 R_A=\mybr{1-\eta_N}S_N\\
R_A&=\frac{\mybr{1-\eta_N} U_{A,N} I_{A,N}}{I_{A,N}^2}=\frac{\mybr{1-\num{0.78}}\SI{440}{\volt}}{\SI{64.10}{\ampere}}=\SI{1.510}{\ohm}
\end{align}

\subsection{}
\begin{align}
P_N&=M_N \omega_N\\
M_N&=\frac{P_N}{2\pi n_N}=\frac{\SI{22}{\kilo\watt}}{2\pi\frac{\SI{1400}{\rounds\per\minute}}{\SI{60}{\second\per\minute}}}=\SI{150.1}{\newton\metre}
\end{align}

\subsection{}
\begin{align}
M_N&=\frac{k_1\phi_N}{2\pi}I_{A,N}\\
k_1\phi_N&=\frac{2\pi M_N}{I_{A,N}}=\frac{2\pi \SI{150.1}{\newton\metre}}{\SI{64.10}{\ampere}}=\SI{14.71}{\volt\second}
\end{align}

\section{}
Hinweis: In der Angabe gibt es einen Fehler, da die Gleichungen mit diesen Parametern kein lösbares Gleichungssystem darstellen. Das sieht man wenn man sich die Drehzahl im Nennpunkt einmal über die Ankerspannungsgleichung
\begin{equation}
n_N=\frac{U_{A,N}-R_A I_{A,N}}{k_1\phi_N}=\frac{\SI{400}{\volt}-\SI{2.5}{\ohm}\cdot\SI{45}{\ampere}}{\SI{16}{\volt\second}}\SI{60}{\second\per\minute}=\SI{1078}{\rounds\per\minute}
\end{equation}
und einmal über die Leistung
\begin{equation}
n_N=\frac{P_N}{2\pi\frac{k_1\phi_N}{2\pi}I_{A,N}}=\frac{\SI{14}{\kilo\watt}}{2\pi\frac{\SI{16}{\volt\second}}{2\pi}\SI{45}{\ampere}}\SI{60}{\second\per\minute}=\SI{1167}{\rounds\per\minute}
\end{equation}
berechnet. Diesen Fehler habe ich bei der weiteren Berechnung ignoriert.
\begin{align}
\num{0.5}P_N&=S-P_V=U_A I_A - I_A^2 R_A\\
0&=I_A^2 R_A - I_A U_A + \num{0.5}P_N\\
I_{A,1,2}&=\frac{U_A\pm\sqrt{U_A^2-4 R_A \num{0.5}P_N}}{2 R_A}\\
&=\frac{\SI{320}{\volt}\pm\sqrt{\mybr{\SI{320}{\volt}}^2-4 \cdot \SI{2.5}{\ohm}\cdot \num{0.5}\cdot\SI{14}{\kilo\watt}}}{2 \cdot \SI{2.5}{\ohm}}\\
I_{A,1}&=\SI{100}{\ampere}\\
I_{A,2}&=\SI{28}{\ampere}
\end{align}
$I_{A,1}$ ist viel größer als der Nennstrom und daher keine sinnvolle Lösung.
\begin{align}
U_A&=R_A I_A + k_1\phi\mybr{I_A} n\\
n_N&=\frac{U_A-R_A I_A}{k_1\phi_N\sqrt{\frac{I_A}{I_{A,N}}}}=\frac{\SI{320}{\volt}-\SI{2.5}{\ohm}\cdot\SI{28}{\ampere}}{\SI{16}{\volt\second}\sqrt{\frac{\SI{28}{\ampere}}{\SI{45}{\ampere}}}}\SI{60}{\second\per\minute}=\SI{1188}{\rounds\per\minute}
\end{align}

\section{}
\subsection{}
\begin{align}
n&=0\\
U_A&=R_A I_A\\
I_A&=\frac{U_A}{R_A}=\frac{\SI{400}{\volt}}{\SI{2.5}{\ohm}}=\SI{160}{\ampere}\\
M&=\frac{k_1\phi\mybr{I_A}}{2\pi}I_A=\frac{k_1\phi_N\sqrt{\frac{I_A}{I_{A,N}}}}{2\pi}I_A\\
&=\frac{\SI{16}{\volt\second}\sqrt{\frac{\SI{160}{\ampere}}{\SI{45}{\ampere}}}}{2\pi}\SI{160}{\ampere}=\SI{768.3}{\newton\metre}
\end{align}

\subsection{}
\begin{figure*}[!h]
\centering
\begin{circuitikz}
\begin{scope}[scale=0.8]
	%\draw [help lines] (-1,-1) grid (7,4); %Zeichnet Raster und vereinfacht damit das Zeichnen
	
	%Schaltung
	\draw (0,3) to[R, l_=$R_V$] (3,3)
	to[R, l_=$R_A$] (6,3)
	to[european voltage source] (6,0)
	-- (0,0);
	
	%Spannungspfeil
	\draw (0,0) to [open, v^=$U_{Netz}$] (0,3);
	\draw (6.5,3) to [open, v^>=$\SI{0}{\volt}$] (6.5,0);
	
	
\end{scope}
\end{circuitikz}
\end{figure*}
\begin{align}
M_N&=\frac{k_1\phi_N}{2\pi}I_{A,N}=\frac{\SI{16}{\volt\second}}{2\pi}\SI{45}{\ampere}=\SI{114.6}{\newton\metre}\\
\num{1.5}M_N&=\frac{k_1\phi_N\frac{I_A^{\frac{3}{2}}}{\sqrt{I_{A,N}}}}{2\pi}\\
I_A&=\sqrt[\frac{3}{2}]{\frac{2\pi\num{1.5}M_N}{k_1\phi_N}\sqrt{I_{A,N}}}=\sqrt[\frac{3}{2}]{\frac{2\pi\num{1.5}\cdot\SI{114.6}{\newton\metre}}{\SI{16}{\volt\second}}\sqrt{\SI{45}{\ampere}}}=\SI{58.97}{\ampere}\\
R_{Ges}&=\frac{U_{Netz}}{I_A}=\frac{U_{A,N}}{I_A}=\frac{\SI{400}{\volt}}{\SI{58.97}{\ampere}}=\SI{6.783}{\ohm}\\
R_V&=R_{Ges}-R_A=\SI{6.783}{\ohm}-\SI{2.5}{\ohm}=\SI{4.283}{\ohm}
\end{align}

\section{}
\begin{figure*}[!ht]
\centering
\begin{circuitikz}
\begin{scope}[scale=0.8]
	%\draw [help lines] (-1,-1) grid (7,4); %Zeichnet Raster und vereinfacht damit das Zeichnen
	
	%Schaltung
	\draw (0,0) to[R, l_=$R_B$] (0,3)
	to[R, l_=$R_A$] (3,3)
	to[european voltage source] (3,0)
	-- (0,0);
	
	%Spannungspfeil
	\draw (3.5,3) to [open, v^>=$U_i$] (3.5,0);
	
	
\end{scope}
\end{circuitikz}
\end{figure*}
\begin{align}
R_{Ges}&=\frac{U_i}{I_A}=\frac{k_1\phi\mybr{I_A}n}{I_A}=\frac{\SI{16}{\volt\second}\sqrt{\frac{\SI{58.97}{\ampere}}{\SI{45}{\ampere}}}\frac{\SI{1500}{\rounds\per\minute}}{\SI{60}{\second\per\minute}}}{\SI{58.97}{\ampere}}=\SI{7.765}{\ampere}\\
R_B&=R_{Ges}-R_A=\SI{7.765}{\ohm}-\SI{2.5}{\ohm}=\SI{5.265}{\ohm}
\end{align}

\section{}
\subsection{}
\begin{align}
U_{A,N}&=R_A I_A+k_1\phi\mybr{I_A}n\\
n&=\frac{U_{A,N}-R_A I_A}{k_1\phi_N\sqrt{\frac{I_A}{I_{A,N}}}}\\
n_N&=\frac{U_{A,N}-R_A I_{A,N}}{k_1\phi_N}\\
\frac{n}{n_N}&=\frac{U_{A,N}-R_A I_A}{\mybr{U_{A,N}-R_A I_{A,N}}\sqrt{\frac{I_A}{I_{A,N}}}}=\frac{\SI{400}{\volt}-\SI{2.5}{\ohm}i_A\SI{45}{\ampere}}{\mybr{\SI{400}{\volt}-\SI{2.5}{\ohm}\cdot\SI{45}{\ampere}}\sqrt{i_A}}\\
M&=\frac{k_1\phi\mybr{I_A}}{2\pi}I_A=\frac{k_1\phi_N\sqrt{\frac{I_A}{I_{A,N}}}}{2\pi}I_A\\
M_N&=\frac{k_1\phi_N}{2\pi}I_{A,N}\\
\frac{M}{M_N}&=\mybr{\frac{I_A}{I_{A,N}}}^{\frac{3}{2}}=\mybr{i_A}^{\frac{3}{2}}
\end{align}
\begin{table}[htbp]
	\begin{center}
	\begin{tabular}{SSS}
		\toprule
		{$i_A$} & {$\frac{n}{n_N}$} & {$\frac{M}{M_N}$} \\
		\midrule
		0.1 & 4.276 & 0.03162 \\
		0.5 & 1.691 & 0.3536 \\
		1.0 & 1.000 & 1.000 \\
		1.5 & 0.6567 & 1.837 \\
		\bottomrule
	\end{tabular}
	\end{center}
\end{table}
\begin{figure*}[!hp]
\centering
\begin{tikzpicture}
	\begin{axis}[
        xlabel=$\frac{n}{n_N}$,
        ylabel=$\frac{M}{M_N}$,
		grid,
		ymin=0]
    \addplot[smooth,mark=*,blue] plot coordinates {
        (4.276 , 0.03162)
		(1.691 , 0.3536)
		(1.000 , 1.000)
		(0.6567 , 1.837)
    };
    \end{axis}
\end{tikzpicture}
\end{figure*}

\clearpage
%%%%%%%%%%%%%%%%%%%%%%%%%%%%%%%%%%%%%%%%%%%%%%%%%%%%%%%%%%%%%%%%%%%%%%%%%%%%%%%%%%%%%%%%%%%%%%%%%%%%
%%%%%%%%%%%%%%%%%%%%%%%%%%%%%%%%%%%%%%%%%%%%%%%%%%%%%%%%%%%%%%%%%%%%%%%%%%%%%%%%%%%%%%%%%%%%%%%%%%%%
\part{2014 GSM}
\section{}
\begin{align}
P_N&=\eta_N U_{A,N} I_{A,N}\\
I_{A,N}&=\frac{P_N}{\eta_N U_{A,N}}=\frac{\SI{25}{\kilo\watt}}{\num{0.82}\cdot\SI{440}{\volt}}=\SI{69.29}{\ampere}\\
P_N&=M_N\omega_N\\
M_N&=\frac{P_N}{\omega_N}=\frac{P_N}{2\pi n_N}=\frac{\SI{25}{\kilo\watt}}{2\pi \frac{\SI{1300}{\rounds\per\minute}}{\SI{60}{\second\per\minute}}}=\SI{183.6}{\newton\metre}\\
&=\frac{k_1\phi}{2\pi}I_{A,N}\\
k_1\phi&=2\pi \frac{M_N}{I_{A,N}}=2\pi\frac{\SI{183.6}{\newton\metre}}{\SI{69.29}{\ampere}}=\SI{16.65}{\volt\second}\\
U_{A,N}&=R_A I_{A,N} + k_1\phi n_N\\
R_A&=\frac{U_{A,N}-k_1\phi n_N}{I_{A,N}}=\frac{\SI{440}{\volt}-\SI{16.65}{\volt\second}\frac{\SI{1300}{\rounds\per\minute}}{\SI{60}{\second\per\minute}}}{\SI{69.29}{\ampere}}=\SI{1.144}{\ohm}
\end{align}

\section{}
\begin{align}
M_N&=\frac{k_1 \phi_N}{2\pi} I_{A,N}\\
k_1\phi_N&=2\pi\frac{M_N}{I_{A,N}}=2\pi\frac{\SI{35}{\newton\metre}}{\SI{20}{\ampere}}=\SI{11.00}{\volt\second}\\
M_L&=\frac{k_1 \phi}{2\pi} I_{A,L}\\
U_A&=R_A I_{A,L}+k_1\phi n_L=R_A \frac{2\pi M_L}{k_1\phi}+k_1\phi n_L\\
0&=\mybr{k_1\phi}^2 n_L - k_1 \phi U_A + 2\pi R_A M_L\\
k_1\phi_{1,2}&=\frac{U_A\pm\sqrt{U_A^2-4n_L 2 \pi R_A M_L}}{2 n_L}\\
&=\frac{\SI{220}{\volt}\pm\sqrt{\mybr{\SI{220}{\volt}}^2-4\frac{\SI{1900}{\rounds\per\minute}}{\SI{60}{\second\per\minute}}2\pi\SI{1}{\ohm}\cdot\SI{15}{\newton\metre}}}{2\frac{\SI{1900}{\rounds\per\minute}}{\SI{60}{\second\per\minute}}}\\
k_1\phi_1&=\SI{6.489}{\volt\second}\\
k_1\phi_2&=\SI{0.4587}{\volt\second}\\
I_{A,1}&=\frac{2\pi M_L}{k_1\phi_1}=\frac{2\pi\SI{15}{\newton\metre}}{\SI{6.489}{\volt\second}}=\SI{14.52}{\ampere}\\
I_{A,2}&=\frac{2\pi M_L}{k_1\phi_2}=\frac{2\pi\SI{15}{\newton\metre}}{\SI{0.4587}{\volt\second}}=\SI{205.5}{\ampere}\\
&\text{$I_{A,2}$ ist 20 mal größer als Nennstrom und daher unrealistisch}\\
\varphi&=\frac{k_1\phi_1}{k_1\phi_N}=\frac{\SI{6.489}{\volt\second}}{\SI{11.00}{\volt\second}}=\num{0.5899}
\end{align}

\section{}
\subsection{}
\begin{align}
k_1\phi_N&=\frac{2\pi M_N}{I_{A,N}}=\frac{2\pi\SI{100}{\newton\metre}}{\SI{60}{\ampere}}=\SI{10.47}{\volt\second}\\
\num{1.5} M_N&=\frac{k_1\phi\mybr{I_A}}{2\pi}I_A=\frac{k_1\phi_N}{2\pi}\mybr{\frac{I_A}{I_{A,N}}}^{\num{0.6}} I_A\\
I_A&=\sqrt[\num{1.6}]{\frac{2\pi \num{1.5}M_N I_{A,N}^{\num{0.6}}}{k_1\phi_N}}=\sqrt[\num{1.6}]{\frac{2\pi \num{1.5}\cdot\SI{100}{\newton\metre} \mybr{\SI{60}{\ampere}}^{\num{0.6}}}{\SI{10.47}{\volt\second}}}=\SI{77.31}{\ampere}
\end{align}
\begin{figure*}[!h]
\centering
\begin{circuitikz}
\begin{scope}[scale=0.8]
	%\draw [help lines] (-1,-1) grid (7,4); %Zeichnet Raster und vereinfacht damit das Zeichnen
	
	%Schaltung
	\draw (0,3) to[R, l_=$R_V$] (3,3)
	to[R, l_=$R_A$] (6,3)
	to[european voltage source] (6,0)
	-- (0,0);
	
	%Spannungspfeil
	\draw (0,0) to [open, v^=$U_{Netz}$] (0,3);
	\draw (6.5,3) to [open, v^>=$\SI{0}{\volt}$] (6.5,0);
	
	
\end{scope}
\end{circuitikz}
\end{figure*}
\begin{align}
R_{Ges}&=\frac{U_{Netz}}{I_A}=\frac{U_{A,N}}{I_A}=\frac{\SI{440}{\volt}}{\SI{77.31}{\ampere}}=\SI{5.691}{\ohm}\\
R_V&=R_{Ges}-R_A=\SI{5.691}{\ohm}-\SI{1}{\ohm}=\SI{4.691}{\ohm}
\end{align}

\subsection{}
\begin{equation}
I_A=\sqrt[\num{1.6}]{\frac{2\pi \num{1.8}M_N I_{A,N}^{\num{0.6}}}{k_1\phi_N}}=\sqrt[\num{1.6}]{\frac{2\pi \num{1.8}\cdot\SI{100}{\newton\metre} \mybr{\SI{60}{\ampere}}^{\num{0.6}}}{\SI{10.47}{\volt\second}}}=\SI{86.65}{\ampere}
\end{equation}
\begin{figure*}[!h]
\centering
\begin{circuitikz}
\begin{scope}[scale=0.8]
	%\draw [help lines] (-1,-1) grid (7,4); %Zeichnet Raster und vereinfacht damit das Zeichnen
	
	%Schaltung
	\draw (0,0) to[R, l_=$R_B$] (0,3)
	to[R, l_=$R_A$] (3,3)
	to[european voltage source] (3,0)
	-- (0,0);
	
	%Spannungspfeil
	\draw (3.5,3) to [open, v^>=$U_i$] (3.5,0);
	
	
\end{scope}
\end{circuitikz}
\end{figure*}
\begin{align}
R_{Ges}&=\frac{U_i}{I_A}=\frac{k_1\phi\mybr{I_A}n}{I_A}=\frac{k_1\phi_N\mybr{\frac{I_A}{I_{A,N}}}^2 n}{I_A}\\
&=\frac{\SI{10.47}{\volt\second}\mybr{\frac{\SI{86.65}{\ampere}}{\SI{60}{\ampere}}}^{\num{0.6}}\frac{\SI{1500}{\rounds\per\minute}}{\SI{60}{\second\per\minute}}}{\SI{86.65}{\ampere}}=\SI{3.766}{\ohm}\\
R_B&=R_{Ges}-R_A=\SI{3.766}{\ohm}-\SI{1}{\ohm}=\SI{2.766}{\ohm}
\end{align}

\subsection{}
\begin{align}
\frac{M}{M_N}&=\frac{k_1\phi\mybr{I_A}I_A}{2\pi M_N}=\frac{k_1\phi_N\frac{I_A^{\num{1.6}}}{I_{A,N}^{\num{0.6}}}}{2\pi M_N}=\frac{\SI{10.47}{\volt\second}\frac{I_A^{\num{1.6}}}{\mybr{\SI{60}{\ampere}}^{\num{0.6}}}}{2\pi\SI{100}{\newton\metre}}\\
U_A&=R_A I_A + k_1 \phi\mybr{I_A} n\\
n_N&=\frac{U_{A,N}-R_A I_{A,N}}{k\phi_N}\\
\frac{n}{n_N}&=\frac{\frac{U_{A,N}-R_A I_A}{k\phi\mybr{I_A}}}{\frac{U_{A,N}-R_A I_{A,N}}{k\phi_N}}\\
&=\frac{U_{A,N}-R_A I_A}{\mybr{\frac{I_A}{I_{A,N}}}^{\num{0.6}}\mybr{U_{A,N}-R_A I_{A,N}}}\\
&=\frac{\SI{440}{\volt}-\SI{1}{\ohm}I_A}{\mybr{\frac{I_A}{\SI{60}{\ampere}}}^{\num{0.6}}\mybr{\SI{440}{\volt}-\SI{1}{\ohm}\cdot\SI{60}{\ampere}}}
\end{align}
\begin{table}[htbp]
	\begin{center}
	\begin{tabular}{SSS}
		\toprule
		{$i_A$} & {$\frac{n}{n_N}$} & {$\frac{M}{M_N}$} \\
		\midrule
		0.1 & 4.547 & 0.02511 \\
		0.25 & 2.569 & 0.1088 \\
		0.5 & 1.635 & 0.3298 \\
		1.0 & 1.000 & 0.9998 \\
		1.5 & 0.7222 & 1.913 \\
		\bottomrule
	\end{tabular}
	\end{center}
\end{table}
\begin{figure*}[!hp]
\centering
\begin{tikzpicture}
	\begin{axis}[
        xlabel=$\frac{n}{n_N}$,
        ylabel=$\frac{M}{M_N}$,
		grid,
		ymin=0]
    \addplot[smooth,mark=*,blue] plot coordinates {
        (4.547 , 0.02511)
		(2.569 , 0.1088)
		(1.635 , 0.3298)
		(1.000 , 0.9998)
		(0.7222 , 1.913)
    };
    \end{axis}
\end{tikzpicture}
\end{figure*}

\section{}
\begin{align}
U_{Netz}&=U_0+R_i I_A = \SI{180}{\volt} + \SI{4}{\ohm}I_A\\
u_L&=L_A\frac{\d i_A}{\d t}=U_i-U_{Netz}\\
\frac{\partial u_L}{\partial i_A}&<0\\
\frac{\partial U_i}{\partial i_A}&<\frac{\partial U_{Netz}}{\partial i_A}
\end{align}
\begin{figure*}[!hp]
\centering
\begin{circuitikz}
\begin{scope}[scale=0.8]
	%\draw [help lines] (-1,-1) grid (7,4); %Zeichnet Raster und vereinfacht damit das Zeichnen
	
	%Schaltung
	\draw (0,3) to[american inductor, l_=$L_A$, i<=$i_A$] (3,3)
	to[european voltage source] (3,0)
	-- (0,0);
	
	%Spannungspfeil
	\draw (0,0) to [open, v^=$U_{Netz}$] (0,3);
	\draw (3.5,3) to [open, v^>=$U_i$] (3.5,0);
	
	
\end{scope}
\end{circuitikz}
\end{figure*}
\begin{figure*}[!hp]
\centering
\begin{tikzpicture}
	\begin{axis}[
        xlabel=$I_A$ in \si{\ampere},
        ylabel=$U$ in \si{\volt},
		grid,
		ymin=0,
		xmin=0,
		ymax=600,
		xmax=100,
		legend to name=stab-legende,
		name=stabplot]
    \addplot[smooth,blue] plot coordinates {
        (0 , 0)
		(6 , 100)
		(10 , 140)
		(17 , 200)
		(20, 225)
		(30, 290)
		(40,350)
		(50,395)
		(60,440)
		(70,480)
		(80,510)
		(90,535)
		(100,550)
    };
	\addlegendentry{$U_A$}
	\addplot[domain=0:100, red] {180+4*x};
	\addlegendentry{$U_{Netz}$}
	\draw[color=black] (axis cs:34.5,318) circle (2pt)  node[below right] {instabil};
	\draw[color=black] (axis cs:87,528) circle (2pt) node[above left] {stabil};
    \end{axis}
	\node at (stabplot.east) [anchor=south, xshift=-1.50cm, yshift=-1.5cm]
			{\pgfplotslegendfromname{stab-legende}};
\end{tikzpicture}
\end{figure*}

\clearpage
%%%%%%%%%%%%%%%%%%%%%%%%%%%%%%%%%%%%%%%%%%%%%%%%%%%%%%%%%%%%%%%%%%%%%%%%%%%%%%%%%%%%%%%%%%%%%%%%%%%%
%%%%%%%%%%%%%%%%%%%%%%%%%%%%%%%%%%%%%%%%%%%%%%%%%%%%%%%%%%%%%%%%%%%%%%%%%%%%%%%%%%%%%%%%%%%%%%%%%%%%
\part{2015 GSM}
\section{}
\subsection{}
\begin{align}
P_N&=M_N\omega_N\\
M_N&=\frac{P_N}{2\pi n_N}=\frac{\SI{20}{\kilo\watt}}{2\pi \frac{\SI{2000}{\rounds\per\minute}}{\SI{60}{\second\per\minute}}}=\SI{95.49}{\newton\metre}\\
P_N&=\eta_N U_{A,N} I_{A,N}\\
I_{A,N}&=\frac{P_N}{\eta_N U_{A,N}}=\frac{\SI{20}{\kilo\watt}}{\num{0.91}\cdot\SI{440}{\volt}}=\SI{49.95}{\ampere}\\
I_{A,N}^2 R_i&=U_{A,N} I_{A,N}\mybr{1-\eta_N}\\
R_i&=\frac{\mybr{1-\eta_N}U_{A,N}I_{A,N}}{I_{A,N}^2}=\frac{\mybr{1-\num{0.91}}\SI{440}{\volt}}{\SI{49.95}{\ampere}}=\SI{0.7928}{\ohm}\\
U_{A,N}&=R_i I_{A,N}+k_1\phi n_N\\
k_1\phi&=\frac{U_{A,N}-R_i I_{A,N}}{n_N}=\frac{\SI{400}{\volt}-\SI{0.7928}{\ohm}\cdot\SI{49.95}{\ampere}}{\frac{\SI{2000}{\rounds\per\minute}}{\SI{60}{\second\per\minute}}}=\SI{12.01}{\volt\second}\\
n_0&=\frac{U_{A,N}}{k_1\phi}=\frac{\SI{440}{\volt}}{\SI{12.01}{\volt\second}}\SI{60}{\second\per\minute}=\SI{2198}{\rounds\per\minute}
\end{align}

\subsection{}
\begin{align}
M_{Last}&=F l=m g l=\SI{2000}{\kilogram}\cdot\SI{9.81}{\meter\per\second\squared}\cdot\SI{15}{\centi\metre}=\SI{2943}{\newton\metre}\\
i&=\frac{n_1}{n_2}=\frac{M_2}{M_1}=\frac{M_{Last}}{M_N}=\frac{\SI{2943}{\newton\metre}}{\SI{95.49}{\newton\metre}}=\num{30.82}
\end{align}
$i=31$ gewählt damit $I_A < I_{A,N}$

\subsection{}
\begin{align}
n_2&=\frac{v}{2\pi r}=\frac{\SI{20}{\metre\per\minute}}{2\pi\SI{15}{\centi\metre}}=\SI{21.22}{\rounds\per\minute}\\
n_1&=n_2 i =\SI{21.22}{\rounds\per\minute}\cdot 31=\SI{657.8}{\rounds\per\minute}\\
I_A&=\frac{2\pi\frac{M_{Last}}{i}}{k_1\phi}=\frac{2\pi\frac{\SI{2943}{\newton\metre}}{31}}{\SI{12.01}{\volt\second}}=\SI{49.67}{\ampere}\\
U_A&=R_i I_A+k_1\phi n_1=\SI{0.7928}{\ohm}\cdot\SI{49.67}{\ampere}+\SI{12.01}{\volt\second}\frac{\SI{657.8}{\rounds\per\minute}}{\SI{60}{\second\per\minute}}=\SI{171.0}{\volt}\\
\alpha&=\arccos\mybr{\frac{U_A}{U_{di\alpha}}}=\arccos\mybr{\frac{\SI{171.0}{\volt}}{\sqrt{2}\frac{\SI{400}{\volt}}{\sqrt{3}}\sqrt{3}\frac{3}{\pi}}}=\SI{71.55}{\degree}
\end{align}

\subsection{}
kein Plan

\section{}
\begin{align}
U_{Netz}&=U_0+R_i I_A = \SI{170}{\volt} + \SI{4}{\ohm}I_A\\
u_L&=L_A\frac{\d i_A}{\d t}=U_i-U_{Netz}\\
\frac{\partial u_L}{\partial i_A}&<0\\
\frac{\partial U_i}{\partial i_A}&<\frac{\partial U_{Netz}}{\partial i_A}
\end{align}
\begin{figure*}[!hp]
	\centering
	\begin{circuitikz}
		\begin{scope}[scale=0.8]
			%\draw [help lines] (-1,-1) grid (7,4); %Zeichnet Raster und vereinfacht damit das Zeichnen
			
			%Schaltung
			\draw (0,3) to[american inductor, l_=$L_A$, i<=$i_A$] (3,3)
			to[european voltage source] (3,0)
			-- (0,0);
			
			%Spannungspfeil
			\draw (0,0) to [open, v^=$U_{Netz}$] (0,3);
			\draw (3.5,3) to [open, v^>=$U_i$] (3.5,0);
			
			
		\end{scope}
	\end{circuitikz}
\end{figure*}
\begin{figure*}[!hp]
	\centering
	\begin{tikzpicture}
	\begin{axis}[
	xlabel=$I_A$ in \si{\ampere},
	ylabel=$U$ in \si{\volt},
	grid,
	ymin=0,
	xmin=0,
	ymax=600,
	xmax=100,
	legend to name=stab-legende,
	name=stabplot]
	\addplot[smooth,blue] plot coordinates {
		(0 , 0)
		(6 , 100)
		(10 , 140)
		(17 , 200)
		(20, 225)
		(30, 290)
		(40,350)
		(50,395)
		(60,440)
		(70,480)
		(80,510)
		(90,535)
		(100,550)
	};
	\addlegendentry{$U_A$}
	\addplot[domain=0:100, red] {170+4*x};
	\addlegendentry{$U_{Netz}$}
	\draw[color=black] (axis cs:30,290) circle (2pt)  node[below right] {instabil};
	\draw[color=black] (axis cs:92,538) circle (2pt) node[above left] {stabil};
	\end{axis}
	\node at (stabplot.east) [anchor=south, xshift=-1.50cm, yshift=-1.5cm]
	{\pgfplotslegendfromname{stab-legende}};
	\end{tikzpicture}
\end{figure*}

\section{}
\subsection{}
\begin{figure*}[!h]
	\centering
	\begin{circuitikz}
		\begin{scope}[scale=0.8]
			%\draw [help lines] (-1,-1) grid (7,7); %Zeichnet Raster und vereinfacht damit das Zeichnen
			
			%Ankerkreis 1
			\draw (0,7) -- (2,7)
			to [short, i>=$I_A$] (2,6);
			
			%Motor
			\draw[fill=black] (1.65,2.85) rectangle (2.35,6.15);
			\draw[fill=white] (2,4.5) ellipse (1.25 and 1.25);
			\draw [->] (3,4.5) -- (1,4.5) node [below right]{$\Phi$};
			\draw [->] (2,5.5) -- (2,3.5) node [above right]{$I_A$};
			\draw [->] (0.5,4.5) arc (180:120:1.5) node [above]{$n$};
			\draw [->] (0,4.5) arc (180:130:2) node [above]{$M$};
			
			%Ankerkreis 2
			\draw (2,3) |- (3,2.5)
			|- (2,-0.5) 
			-- (2,0) node[right] {B1}
			to [L] (2,2) node[right] {B2}
			-| (1,0)
			-- (0,0);
			
			%Erregerkreis
			\draw (6,6) to [short, i>=$I_E$] (6,4.5) node [below] {F1}
			to [L] (4,4.5) node [below] {F2}
			-- (4,6);
			
			%Spannungspfeil
			{
				\ctikzset{voltage/distance from node=.2}
				\draw (-0.5,0) to [open, v^=$U_A$] (-0.5,7);
			}
			\draw (4,6) to [open, v^=$U_E$] (6,6);
			
		\end{scope}
	\end{circuitikz}
\end{figure*}

\subsection{}
\begin{align}
M_N&=\frac{k_1\phi}{2\pi}I_{A,N}\\
k_1\phi&=\frac{2\pi M_N}{I_{A,N}}=\frac{2\pi\SI{20}{\newton\metre}}{\SI{10}{\ampere}}=\SI{12.57}{\volt\second}\\
R_A I_{A,N}^2&=\mybr{1-\eta_N}U_{A,N}I_{A,N}\\
R_A&=\frac{\mybr{1-\eta_N}U_{A,N}}{I_{A,N}}=\frac{\mybr{1-\num{0.85}\SI{440}{\volt}}}{\SI{10}{\ampere}}=\SI{6.6}{\ohm}\\
\num{0.7}M_N&\rightarrow\num{0.7}{I_{A,N}}\\
\num{0.5}U_{A,N}&=R_A\num{0.7}I_{A,N}+k_1\phi n\\
n&=\frac{\num{0.5}\cdot U_{A,N}-R_A\num{0.7}I_{A,N}}{k_1\phi}\\
&=\frac{\num{0.5}\cdot\SI{440}{\volt}-\SI{6.6}{\ohm}\cdot\num{0.7}\cdot\SI{10}{\ampere}}{\SI{12.57}{\volt\second}}\SI{60}{\second\per\minute}=\SI{829.6}{\rounds\per\minute}
\end{align}

\subsection{}
\begin{align}
U_{A,N}&=R_A I_{A,N}+k_1\phi n_N\\
n_N&=\frac{U_{A,N}-R_A I_{A,N}}{k_1\phi}=\frac{\SI{440}{\volt}-\SI{6.6}{\ohm}\cdot\SI{10}{\ampere}}{\SI{12.57}{\volt\second}}\SI{60}{\second\per\minute}=\SI{1785}{\rounds\per\minute}
\end{align}

Ab $\SI{1785}{\rounds\per\minute}$ erfolgt Feldschwächung.

\begin{figure*}[!ht]
	\centering
	\begin{tikzpicture}
	\begin{axis}[
	xlabel=$n$ in \si{\rounds\per\minute},
	grid,
	ymin=0,
	xmin=0,
	ymax=2000,
	xmax=3000,
	yticklabels={,,},
	legend to name=qualitativ-legende,
	name=qualitativplot]
	
	\addplot[domain=0:1785, blue] {x};
	\addlegendentry{$P_{el}$}
	\addplot[domain=0:1785, red] {1500};
	\addlegendentry{$M$}
	\addplot[domain=0:1785, green] {0.8*x};
	\addlegendentry{$U_A$}
	\addplot[domain=0:3000, yellow] {1300};
	\addlegendentry{$I_A$}
	\addplot[domain=0:1785, brown] {1200};
	\addlegendentry{$I_E$}
	
	\addplot[domain=1785:3000, blue] {1785};
	\addplot[domain=1785:3000, red] {1500*1785/x};
	\addplot[domain=1785:3000, green] {0.8*1785};
	\addplot[domain=1785:3000, brown] {1200*1785/x};
	
	\end{axis}
	\node at (qualitativplot.east) [anchor=south, xshift=-2.90cm, yshift=-2.7cm]
	{\pgfplotslegendfromname{qualitativ-legende}};
	\end{tikzpicture}
\end{figure*}

\subsection{}
\begin{equation}
M_N=\frac{k_1\phi\mybr{1-\num{0.05}\mybr{\frac{I_{A,N}}{I_{A,N}}}^2}}{2\pi}I_{A,N}=\frac{\SI{12.57}{\volt\second}\mybr{1-\num{0.05}}}{2\pi}\SI{10}{\ampere}=\SI{19.01}{\newton\metre}
\end{equation}

\subsection{}
\begin{equation}
n_N=\frac{U_{A,N}-R_A I_{A,N}}{k_1\phi\mybr{1-\num{0.05}\mybr{\frac{I_{A,N}}{I_{A,N}}}^2}}=\frac{\SI{440}{\volt}-\SI{6.6}{\ohm}\cdot\SI{10}{\ampere}}{\SI{12.57}{\volt\second}\mybr{1-\num{0.05}}}\SI{60}{\second\per\minute}=\SI{1879}{\rounds\per\minute}
\end{equation}

\subsection{}
\begin{equation}
n=\frac{\num{0.5}U_{A,N}-R_A\num{0.5} I_{A,N}}{k_1\phi\mybr{1-\num{0.05}\mybr{\frac{\num{0.5}I_{A,N}}{I_{A,N}}}^2}}=\frac{\num{0.5}\cdot\SI{440}{\volt}-\SI{6.6}{\ohm}\cdot\num{0.5}\cdot\SI{10}{\ampere}}{\SI{12.57}{\volt\second}\mybr{1-\num{0.05}\cdot\num{0.5}^2}}\SI{60}{\second\per\minute}=\SI{903.9}{\rounds\per\minute}
\end{equation}

\clearpage
%%%%%%%%%%%%%%%%%%%%%%%%%%%%%%%%%%%%%%%%%%%%%%%%%%%%%%%%%%%%%%%%%%%%%%%%%%%%%%%%%%%%%%%%%%%%%%%%%%%%
%%%%%%%%%%%%%%%%%%%%%%%%%%%%%%%%%%%%%%%%%%%%%%%%%%%%%%%%%%%%%%%%%%%%%%%%%%%%%%%%%%%%%%%%%%%%%%%%%%%%
\part{2007 PSM}
\section{}
\begin{align}
	m_{R,ist}&=\Psi_M i_q=1\cdot\Im\mybr{\num{0.5}\e^{\j\SI{15}{\degree}}}=1\cdot\num{0.5}\sin\mybr{\SI{15}{\degree}}=\num{0.1294}
\end{align}
\section{}
\begin{align}
	\isab&=\isdq\e^{\j\gamma_m}=\num{0.5}\e^{\j\SI{15}{\degree}}\e^{\j\SI{45}{\degree}}=\num{0.5}\e^{\j\SI{60}{\degree}}\\
	I_{Bez}&=\sqrt{2}I_{Nenn,Str}=\sqrt{2}\cdot\SI{7}{\ampere}=\SI{9.899}{\ampere}\\
	I_1&=I_{Bez}\Re\mybr{\isab}=\SI{9.899}{\ampere}\cdot\Re\mybr{\num{0.5}\e^{\j\SI{60}{\degree}}}=\SI{2.475}{\ampere}\\
	I_2&=I_{Bez}\Re\mybr{\isab\e^{-\j\SI{120}{\degree}}}=\SI{9.899}{\ampere}\cdot\Re\mybr{\num{0.5}\e^{\j\SI{60}{\degree}}\e^{-\j\SI{120}{\degree}}}=\SI{2.475}{\ampere}\\
	I_3&=I_{Bez}\Re\mybr{\isab\e^{-\j\SI{240}{\degree}}}=\SI{9.899}{\ampere}\cdot\Re\mybr{\num{0.5}\e^{\j\SI{60}{\degree}}\e^{-\j\SI{240}{\degree}}}=\SI{-4.950}{\ampere}
\end{align}

\section{}
\begin{align}
	i_d&=0\quad\quad\text{um Kupferverluste zu minimieren}\\
	m_{R,soll}&=\Psi_M i_q\\
	i_q&=\frac{m_{R,soll}}{\Psi_M}=\frac{\num{0.5}}{1}=\num{0.5}\\
	\i_{S,opt,\alpha\beta}&=\i_{S,opt,dq}\e^{\j\gamma_m}=\j\num{0.5}\e^{\j\SI{45}{\degree}}=\num{0.5}\e^{\j\SI{135}{\degree}}=-\num{0.3536}+\j\num{0.3536}
\end{align}

\section{}
\begin{align}
	\u_S&=r_S\i_S+\frac{\d\PPsi_S}{\d\tau}+\j\omega_K\PPsi_S\\
	\psab&=\Psi_M\e^{\j\gamma_m}+l_S\isab\\
	\frac{\d\psab}{\d\tau}&=0+l_S\frac{\d\isab}{\d\tau}\\
	\omega_K&=0\\
	\u_S&=r_S\i_S+l_S\frac{\d\isab}{\d\tau}\\
	&\rotatebox[origin=c]{270}{\laplace}\nonumber\\
	\frac{\u_{S0}}{s}&=r_S\I_S\mybr{s}+s l_S\I_S\mybr{s}-l_S \i_{S0}\\
	\I_S\mybr{s}&=\frac{\u_{S0}}{s\mybr{r_S+l_S s}}+\frac{l_S \i_{S0}}{r_S+l_S s}\\
	&=\u_{S0}\mybr{\frac{1}{r_S s}-\frac{l_S}{r_S\mybr{r_S+l_S s}}}+\frac{l_S \i_{S0}}{r_S+l_S s}\\
	&=\frac{\u_{S0}}{r_S}\mybr{\frac{1}{s}-\frac{1}{\mybr{\frac{r_S}{l_S}+ s}}}+\frac{\i_{S0}}{\frac{r_S}{l_S}+ s}\\
	&\rotatebox[origin=c]{270}{\Laplace}\nonumber\\
	\i_S\mybr{\tau}&=\mybr{\frac{\u_{S0}}{r_S}\mybr{1-\e^{-\frac{r_S}{l_S}\tau}}+\i_{S0}\e^{-\frac{r_S}{l_S}\tau}}\sigma\mybr{\tau}
\end{align}

\section{}
\begin{align}
	n_N&=\frac{f_{Nenn}\SI{60}{\second\per\minute}}{p}\\
	f_{Nenn}&=\frac{p n_N}{\SI{60}{\second\per\minute}}=\frac{2\cdot\SI{3000}{\rounds\per\minute}}{\SI{60}{\second\per\minute}}=\SI{100}{\hertz}\\
	T_{Bez}&=\frac{1}{2\pi f_{Nenn}}=\frac{1}{2\pi\SI{100}{\hertz}}=\SI{1.592}{\milli\second}\\
	I_2\mybr{t}&=I_{Bez}\Re\mybr{\i_S\mybr{\frac{t}{T_{Bez}}}\e^{-\j\SI{120}{\degree}}}\\
	&=\SI{9.899}{\ampere}\cdot\Re\mybr{\mybr{\frac{\num{0.1}\e^{\j\SI{120}{\degree}}}{r_S}\mybr{1-\e^{-\frac{r_S}{l_S}\frac{t}{T_{Bez}}}}+\num{0.5}\e^{\j\SI{60}{\degree}}\e^{-\frac{r_S}{l_S}\frac{t}{T_{Bez}}}}\sigma\mybr{\frac{t}{T_{Bez}}}\e^{-\j\SI{120}{\degree}}}\\
	&=\SI{9.899}{\ampere}\cdot\Re\mybr{\mybr{\frac{\num{0.1}}{r_S}\mybr{1-\e^{-\frac{r_S}{l_S}\frac{t}{T_{Bez}}}}+\num{0.5}\e^{-\j\SI{60}{\degree}}\e^{-\frac{r_S}{l_S}\frac{t}{T_{Bez}}}}\sigma\mybr{t}}\\
	&=\SI{9.899}{\ampere}\mybr{\frac{\num{0.1}}{r_S}\mybr{1-\e^{-\frac{r_S}{l_S}\frac{t}{T_{Bez}}}}-\num{0.25}\e^{-\frac{r_S}{l_S}\frac{t}{T_{Bez}}}}\sigma\mybr{t}\\
\end{align}
Aus der Messkurve kann man $I_2\mybr{t\rightarrow\infty}=\SI{25}{\ampere}$ und $\tau_{mess}=\SI{10}{\milli\second}$ ablesen.
\begin{figure*}[!hp]
	\centering
	\begin{tikzpicture}
	\begin{axis}[
	xlabel=$t$ in \si{\milli\second},
	ylabel=$I_2$ in \si{\ampere},
	grid,
	ymin=0,
	xmin=-5,
	ymax=30,
	xmax=70,]
	\addplot[domain=-5:0, blue] {0.25*9.899};
	\addplot[domain=0:70, blue] {9.899*(0.1/0.03960*(1-e^(-x/10))+0.25*e^(-x/10))};
	\addplot[domain=0:10, black] {9.899*(0.1/0.03960/10-0.25/10)*x+0.25*9.899};
	\draw [decorate,decoration={brace,amplitude=5pt},xshift=0pt,yshift=4pt]
	(axis cs:0,25) -- (axis cs:10,25) node [black,midway,yshift=10pt] 
	{\footnotesize $\tau_{mess}$};
	\end{axis}
	\end{tikzpicture}
\end{figure*}
\begin{align}
	I_2\mybr{t\rightarrow\infty}&=\SI{9.899}{\ampere}\frac{\num{0.1}}{r_S}\\
	r_S&=\frac{\SI{9.899}{\ampere}\cdot\num{0.1}}{\SI{25}{\ampere}}=\num{0.03960}\\
	\tau_{mess}&=T_{Bez}\frac{l_S}{r_S}\\
	l_S&=\frac{\tau_{mess}r_S}{T_{Bez}}=\frac{\SI{10}{\milli\second}\cdot\num{0.03960}}{\SI{1.592}{\milli\second}}=\num{0.2487}
\end{align}

\clearpage
%%%%%%%%%%%%%%%%%%%%%%%%%%%%%%%%%%%%%%%%%%%%%%%%%%%%%%%%%%%%%%%%%%%%%%%%%%%%%%%%%%%%%%%%%%%%%%%%%%%%
%%%%%%%%%%%%%%%%%%%%%%%%%%%%%%%%%%%%%%%%%%%%%%%%%%%%%%%%%%%%%%%%%%%%%%%%%%%%%%%%%%%%%%%%%%%%%%%%%%%%
\part{2009 PSM}
\section{}
\begin{align}
I_{Bez}&=\sqrt{2}I_{N,Str}=\sqrt{2}\cdot\SI{5}{\ampere}=\SI{7.071}{\ampere}\\
\i_S&=\frac{2}{3I_{Bez}}\mybr{I_1+I_2\e^{\j\SI{120}{\degree}}+I_3\e^{\j\SI{240}{\degree}}}\\
&=\frac{2}{3\cdot\SI{7.071}{\ampere}}\mybr{\SI{-2.12}{\ampere}+\SI{4.24}{\ampere}\e^{\j\SI{120}{\degree}}-\SI{2.12}{\ampere}\e^{\j\SI{240}{\degree}}}\\
&=\num{-0.2998}+\j\num{0.5193}\\
\isdq&=\isab\e^{-\j\gamma_m}=\mybr{\num{-0.2998}+\j\num{0.5193}}\e^{-\j\SI{25}{\degree}}\\
&=\num{-0.05225}+\j\num{0.5973}\\
m&=\Psi_M i_q=1\cdot\num{0.5973}=\num{0.5973}
\end{align}

\section{}
\begin{align}
m&=\Psi_M i_q\\
i_q&=\frac{m}{\Psi_M}=\frac{\num{0.7}}{1}=\num{0.7}\\
i_d&=0\quad\quad\text{um Kupferverluste zu minimieren}\\
\isab&=\isdq\e^{\j\gamma_m}=\j\num{0.7}\e^{\j\SI{25}{\degree}}=\num{0.7}\e^{\j\SI{115}{\degree}}=\num{-0.2958}+\j\num{0.6344}
\end{align}

\section{}
\subsection{}
\begin{equation}
\gamma=\gamma_0+\omega_m\Delta\tau=\SI{25}{\degree}+\num{0.3}\cdot\num{0.1}\frac{\SI{180}{\degree}}{\pi}=\SI{26.72}{\degree}
\end{equation}
\subsection{}
\begin{align}
\usab&=r_S\isab+\frac{\d\psab}{\d\tau}+\j\omega_K\psab\\
\omega_K&=0\\
r_S&=0\\
\psab&=\Psi_M\e^{\j\gamma_m}+l_S\isab\\
\frac{\d\psab}{\d\tau}&=\j\omega_m\Psi_M\e^{\j\mybr{\gamma_m\Delta\tau+\gamma_0}}+l_S\frac{\d\isab}{\d\tau}\\
\usab&=\j\omega_m\Psi_M\e^{\j\mybr{\gamma_m\Delta\tau+\gamma_0}}+l_S\frac{\d\isab}{\d\tau}\\
&\rotatebox[origin=c]{270}{\laplace}\nonumber\\
-\frac{1}{s}&=\j\omega_m\Psi_M\e^{\j\gamma_0}\frac{1}{s-\j\omega_m}+l_S\I_S\mybr{s}s-l_S\i_S\mybr{\tau_0}\\
\I_S\mybr{s}&=-\frac{1}{l_S s^2}-\frac{\j\omega_m\Psi_M\e^{\j\gamma_0}}{l_S s\mybr{s-\j\omega_m}}+\frac{\i_S\mybr{\tau_0}}{s}\\
&=-\frac{1}{l_S s^2}-\frac{\j\omega_m\Psi_M\e^{\j\gamma_0}}{l_S}\mybr{-\frac{1}{\j\omega_m s}+\frac{1}{\j\omega_m\mybr{s-\j\omega_m}}}+\frac{\i_S\mybr{\tau_0}}{s}\\
&=-\frac{1}{l_S s^2}+\frac{\Psi_M\e^{\j\gamma_0}}{l_S s}-\frac{\Psi_M\e^{\j\gamma_0}}{l_S\mybr{s-\j\omega_m}}+\frac{\i_S\mybr{\tau_0}}{s}\\
&\rotatebox[origin=c]{270}{\Laplace}\nonumber\\
\i_S\mybr{\Delta\tau}&=\mybr{-\frac{1}{l_S}\Delta\tau+\frac{\Psi_M\e^{\j\gamma_0}}{l_S}\mybr{1-\e^{\j\omega_m\Delta\tau}}+\i_S\mybr{\tau_0}}\sigma\mybr{\Delta\tau}\\
\i_S\mybr{\num{0.1}}&=-\frac{\num{0.1}}{\num{0.3}}+\frac{1\cdot\e^{\j\SI{25}{\degree}}}{\num{0.3}}\mybr{1-\e^{\j\num{0.3}\cdot\num{0.1}\cdot\frac{\SI{180}{\degree}}{\pi}}}-\num{0.2998}+\j\num{0.5193}\\
&=\num{-0.5895}+\j\num{0.4293}
\end{align}

\clearpage
%%%%%%%%%%%%%%%%%%%%%%%%%%%%%%%%%%%%%%%%%%%%%%%%%%%%%%%%%%%%%%%%%%%%%%%%%%%%%%%%%%%%%%%%%%%%%%%%%%%%
%%%%%%%%%%%%%%%%%%%%%%%%%%%%%%%%%%%%%%%%%%%%%%%%%%%%%%%%%%%%%%%%%%%%%%%%%%%%%%%%%%%%%%%%%%%%%%%%%%%%
\part{2010 PSM}
\section{}
\begin{align}
m_{R,ist}&=\Psi_M i_q=1\cdot\Im\mybr{\num{0.65}\e^{\j\SI{50}{\degree}}}=1\cdot\num{0.65}\sin\mybr{\SI{50}{\degree}}=\num{0.4979}
\end{align}
\section{}
\begin{align}
\isab&=\isdq\e^{\j\gamma_m}=\num{0.65}\e^{\j\SI{50}{\degree}}\e^{-\j\SI{30}{\degree}}=\num{0.65}\e^{\j\SI{20}{\degree}}\\
I_{Bez}&=\sqrt{2}I_{Nenn,Str}=\sqrt{2}\cdot\SI{5}{\ampere}=\SI{7.071}{\ampere}\\
I_1&=I_{Bez}\Re\mybr{\isab}=\SI{7.071}{\ampere}\cdot\Re\mybr{\num{0.65}\e^{\j\SI{20}{\degree}}}=\SI{4.319}{\ampere}\\
I_2&=I_{Bez}\Re\mybr{\isab\e^{-\j\SI{120}{\degree}}}=\SI{7.071}{\ampere}\cdot\Re\mybr{\num{0.65}\e^{\j\SI{20}{\degree}}\e^{-\j\SI{120}{\degree}}}=\SI{-0.7981}{\ampere}\\
I_3&=I_{Bez}\Re\mybr{\isab\e^{-\j\SI{240}{\degree}}}=\SI{7.071}{\ampere}\cdot\Re\mybr{\num{0.65}\e^{\j\SI{20}{\degree}}\e^{-\j\SI{240}{\degree}}}=\SI{-3.521}{\ampere}
\end{align}

\section{}
\begin{align}
i_d&=0\quad\quad\text{um Kupferverluste zu minimieren}\\
m_{R,soll}&=\Psi_M i_q\\
i_q&=\frac{m_{R,soll}}{\Psi_M}=\frac{\num{0.5}}{1}=\num{0.5}\\
\i_{S,opt,\alpha\beta}&=\i_{S,opt,dq}\e^{\j\gamma_m}=\j\num{0.5}\e^{-\j\SI{30}{\degree}}=\num{0.5}\e^{\j\SI{60}{\degree}}=\num{0.2500}+\j\num{0.4330}
\end{align}

\section{}
\begin{align}
\u_S&=r_S\i_S+\frac{\d\PPsi_S}{\d\tau}+\j\omega_K\PPsi_S\\
\psab&=\Psi_M\e^{\j\gamma_m}+l_S\isab\\
\frac{\d\psab}{\d\tau}&=0+l_S\frac{\d\isab}{\d\tau}\\
\omega_K&=0\\
\u_S&=r_S\i_S+l_S\frac{\d\isab}{\d\tau}\\
&\rotatebox[origin=c]{270}{\laplace}\nonumber\\
\frac{\u_{S0}}{s}&=r_S\I_S\mybr{s}+s l_S\I_S\mybr{s}-l_S \i_{S0}\\
\I_S\mybr{s}&=\frac{\u_{S0}}{s\mybr{r_S+l_S s}}+\frac{l_S \i_{S0}}{r_S+l_S s}\\
&=\u_{S0}\mybr{\frac{1}{r_S s}-\frac{l_S}{r_S\mybr{r_S+l_S s}}}+\frac{l_S \i_{S0}}{r_S+l_S s}\\
&=\frac{\u_{S0}}{r_S}\mybr{\frac{1}{s}-\frac{1}{\mybr{\frac{r_S}{l_S}+ s}}}+\frac{\i_{S0}}{\frac{r_S}{l_S}+ s}\\
&\rotatebox[origin=c]{270}{\Laplace}\nonumber\\
\i_S\mybr{\tau}&=\mybr{\frac{\u_{S0}}{r_S}\mybr{1-\e^{-\frac{r_S}{l_S}\tau}}+\i_{S0}\e^{-\frac{r_S}{l_S}\tau}}\sigma\mybr{\tau}
\end{align}

\section{}
\begin{align}
n_N&=\frac{f_{Nenn}\SI{60}{\second\per\minute}}{p}\\
f_{Nenn}&=\frac{p n_N}{\SI{60}{\second\per\minute}}=\frac{2\cdot\SI{4000}{\rounds\per\minute}}{\SI{60}{\second\per\minute}}=\SI{133.3}{\hertz}\\
T_{Bez}&=\frac{1}{2\pi f_{Nenn}}=\frac{1}{2\pi\SI{133.3}{\hertz}}=\SI{1.194}{\milli\second}\\
I_3\mybr{t}&=I_{Bez}\Re\mybr{\i_S\mybr{\frac{t}{T_{Bez}}}\e^{-\j\SI{240}{\degree}}}\\
&=\SI{7.071}{\ampere}\cdot\Re\mybr{\mybr{\frac{\num{0.1}\e^{\j\SI{240}{\degree}}}{r_S}\mybr{1-\e^{-\frac{r_S}{l_S}\frac{t}{T_{Bez}}}}+\num{0.65}\e^{\j\SI{20}{\degree}}\e^{-\frac{r_S}{l_S}\frac{t}{T_{Bez}}}}\sigma\mybr{\frac{t}{T_{Bez}}}\e^{-\j\SI{240}{\degree}}}\\
&=\SI{7.071}{\ampere}\cdot\Re\mybr{\mybr{\frac{\num{0.1}}{r_S}\mybr{1-\e^{-\frac{r_S}{l_S}\frac{t}{T_{Bez}}}}+\num{0.65}\e^{-\j\SI{220}{\degree}}\e^{-\frac{r_S}{l_S}\frac{t}{T_{Bez}}}}\sigma\mybr{t}}\\
&=\SI{7.071}{\ampere}\mybr{\frac{\num{0.1}}{r_S}\mybr{1-\e^{-\frac{r_S}{l_S}\frac{t}{T_{Bez}}}}-\num{0.4979}\e^{-\frac{r_S}{l_S}\frac{t}{T_{Bez}}}}\sigma\mybr{t}\\
\end{align}
Aus der Messkurve kann man $I_3\mybr{t\rightarrow\infty}=\SI{15}{\ampere}$ und $\tau_{mess}=\SI{10}{\milli\second}$ ablesen.
\begin{figure*}[!hp]
	\centering
	\begin{tikzpicture}
	\begin{axis}[
	xlabel=$t$ in \si{\milli\second},
	ylabel=$I_3$ in \si{\ampere},
	grid,
	ymin=-5,
	xmin=-5,
	ymax=20,
	xmax=70,]
	\addplot[domain=-5:0, blue] {-0.4979*7.071};
	\addplot[domain=0:70, blue] {7.071*(0.1/0.04714*(1-e^(-x/10))-0.4979*e^(-x/10))};
	\addplot[domain=0:10, black] {7.071*(0.1/0.04714/10+0.4979/10)*x-0.4979*7.071};
	\draw [decorate,decoration={brace,amplitude=5pt},xshift=0pt,yshift=4pt]
	(axis cs:0,15) -- (axis cs:10,15) node [black,midway,yshift=10pt] 
	{\footnotesize $\tau_{mess}$};
	\end{axis}
	\end{tikzpicture}
\end{figure*}
\begin{align}
I_3\mybr{t\rightarrow\infty}&=\SI{7.071}{\ampere}\frac{\num{0.1}}{r_S}\\
r_S&=\frac{\SI{7.071}{\ampere}\cdot\num{0.1}}{\SI{15}{\ampere}}=\num{0.04714}\\
\tau_{mess}&=T_{Bez}\frac{l_S}{r_S}\\
l_S&=\frac{\tau_{mess}r_S}{T_{Bez}}=\frac{\SI{10}{\milli\second}\cdot\num{0.04714}}{\SI{1.194}{\milli\second}}=\num{0.3948}
\end{align}


\clearpage
%%%%%%%%%%%%%%%%%%%%%%%%%%%%%%%%%%%%%%%%%%%%%%%%%%%%%%%%%%%%%%%%%%%%%%%%%%%%%%%%%%%%%%%%%%%%%%%%%%%%
%%%%%%%%%%%%%%%%%%%%%%%%%%%%%%%%%%%%%%%%%%%%%%%%%%%%%%%%%%%%%%%%%%%%%%%%%%%%%%%%%%%%%%%%%%%%%%%%%%%%
\part{2011 PSM}
\section{}
\begin{align}
I_{Bez}&=\sqrt{2}I_{Nenn,Str}=\sqrt{2}I_N=\sqrt{2}\cdot\SI{10}{\ampere}=\SI{14.14}{\ampere}\\
\isab&=\frac{2}{3I_{Bez}}\mybr{I_1+I_2\e^{\j\SI{120}{\degree}}+I_3\e^{\j\SI{240}{\degree}}}\\
&=\frac{2}{3\cdot\SI{14.14}{\ampere}}\mybr{\SI{-13.95}{\ampere}+\SI{9.09}{\ampere}\e^{\j\SI{120}{\degree}}+\SI{4.84}{\ampere}\e^{\j\SI{240}{\degree}}}\\
&=\num{-0.9861}+\j\num{0.1735}\\
\isdq&=\isab\e^{-\j\gamma_m}=\mybr{\num{-0.9861}+\j\num{0.1735}}\e^{-\SI{60}{\degree}}=\num{-0.3428}+\j\num{0.9407}
\end{align}

\section{}
\begin{align}
m_{R,ist}&=\Psi_M i_q=1\cdot\num{0.9407}=\num{0.9407}
\end{align}

\section{}
\begin{align}
i_d&=0\quad\quad\text{um Kupferverluste zu minimieren}\\
m_{R,soll}&=\Psi_M i_q\\
i_q&=\frac{m_{R,soll}}{\Psi_M}=\frac{1}{1}=1\\
\isab&=\isdq\e^{\j\gamma_m}=\j 1\e^{\j\SI{60}{\degree}}=\e^{\j\SI{150}{\degree}}
\end{align}

\section{}
\begin{align}
\u_S&=\frac{2}{3}\mybr{u_{zk}+u_{zk}\e^{\j\SI{120}{\degree}}+0\e^{\j\SI{240}{\degree}}}\\
&=\frac{2}{3}\mybr{\num{1.5}+\num{1.5}\e^{\j\SI{120}{\degree}}+0\e^{\j\SI{240}{\degree}}}\\
&=\num{0.500}+\j\num{0.8660}\\
\PPsi_S&=\Psi_M\e^{\j\gamma_m}+l_S\i_S\\
\frac{\d\psab}{\d\tau}&=\j\omega_m\Psi_M\e^{\j\gamma_m}+l_S\frac{\d\isab}{\d\tau}\\
\omega_K&=0\quad r_S=0\\
\usab&=r_S\isab+\frac{\d\psab}{\d\tau}+\j\omega_K\psab\\
&=\j\omega_m\Psi_M\e^{\j\omega_m}+l_S\frac{\d\isab}{\d\tau}\\
\frac{\d\isab}{\d\tau}&=\frac{\usab-\j\omega_m\Psi_M\e^{\j\gamma_m}}{l_S}\\
&=\frac{\num{0.5000}+\j\num{0.8660}-\j\num{0.4}\cdot 1\e^{\j\SI{60}{\degree}}}{\num{0.3}}\\
&=\num{2.821}+\j\num{2.220}
\end{align}

\section{}
\begin{align}
\isab\mybr{\tau_0+\Delta\tau}&=\isab\mybr{\tau_0}+\frac{\d\isab}{\d\tau}\Delta\tau\\
&=\num{-0.9861}+\j\num{0.1735}+\mybr{\num{2.821}+\j\num{2.220}}\num{0.05}\\
&=\num{-0.8451}+\j\num{0.2845}\\
\gamma_m\mybr{\tau_0+\Delta\tau}&=\gamma_m\mybr{\tau_0}+\omega_m\Delta\tau\\
&=\SI{60}{\degree}+\num{0.4}\cdot\num{0.05}\frac{\SI{180}{\degree}}{\pi}\\
&=\SI{61.15}{\degree}\\
\isdq\mybr{\tau_0+\Delta\tau}&=\isab\mybr{\tau_0+\Delta\tau}\e^{-\j\gamma_m}\\
&=\mybr{\num{-0.8451}+\j\num{0.2845}}\e^{-\j\SI{61.15}{\degree}}\\
&=\num{-0.1586}+\j\num{0.8775}\\
m_{R,ist}\mybr{\tau_0+\Delta\tau}&=\Psi_M i_q\mybr{\tau_0+\Delta\tau}=1\cdot\num{0.8775}=\num{0.8775}
\end{align}

\clearpage

%%%%%%%%%%%%%%%%%%%%%%%%%%%%%%%%%%%%%%%%%%%%%%%%%%%%%%%%%%%%%%%%%%%%%%%%%%%%%%%%%%%%%%%%%%%%%%%%%%%%
%%%%%%%%%%%%%%%%%%%%%%%%%%%%%%%%%%%%%%%%%%%%%%%%%%%%%%%%%%%%%%%%%%%%%%%%%%%%%%%%%%%%%%%%%%%%%%%%%%%%
\part{2013 PSM}
\section{}
\begin{align}
m_{R,ist}&=\Psi_M i_q=1\cdot\Im\mybr{\num{0.65}\e^{\j\SI{50}{\degree}}}=1\cdot\num{0.65}\sin\mybr{\SI{50}{\degree}}=\num{0.4979}
\end{align}
\section{}
\begin{align}
\isab&=\isdq\e^{\j\gamma_m}=\num{0.65}\e^{\j\SI{50}{\degree}}\e^{-\j\SI{30}{\degree}}=\num{0.65}\e^{\j\SI{20}{\degree}}\\
I_{Bez}&=\sqrt{2}I_{Nenn,Str}=\sqrt{2}\SI{5}{\ampere}=\SI{7.071}{\ampere}\\
I_1&=I_{Bez}\Re\mybr{\isab}=\SI{7.071}{\ampere}\cdot\Re\mybr{\num{0.65}\e^{\j\SI{20}{\degree}}}=\SI{4.319}{\ampere}\\
I_2&=I_{Bez}\Re\mybr{\isab\e^{-\j\SI{120}{\degree}}}=\SI{7.071}{\ampere}\cdot\Re\mybr{\num{0.65}\e^{\j\SI{20}{\degree}}\e^{-\j\SI{120}{\degree}}}=\SI{-0.7981}{\ampere}\\
I_3&=I_{Bez}\Re\mybr{\isab\e^{-\j\SI{240}{\degree}}}=\SI{7.071}{\ampere}\cdot\Re\mybr{\num{0.65}\e^{\j\SI{20}{\degree}}\e^{-\j\SI{240}{\degree}}}=\SI{-3.521}{\ampere}
\end{align}

\section{}
\begin{align}
i_d&=0\quad\quad\text{um Kupferverluste zu minimieren}\\
m_{R,soll}&=\Psi_M i_q\\
i_q&=\frac{m_{R,soll}}{\Psi_M}=\frac{\num{0.75}}{1}=\num{0.75}\\
\i_{S,opt,\alpha\beta}&=\i_{S,opt,dq}\e^{\j\gamma_m}=\j\num{0.75}\e^{-\j\SI{30}{\degree}}=\num{0.75}\e^{\j\SI{60}{\degree}}=\num{0.3750}+\j\num{0.6495}
\end{align}

\section{}
\begin{align}
\u_S&=r_S\i_S+\frac{\d\PPsi_S}{\d\tau}+\j\omega_K\PPsi_S\\
\psab&=\Psi_M\e^{\j\gamma_m}+l_S\isab\\
\frac{\d\psab}{\d\tau}&=0+l_S\frac{\d\isab}{\d\tau}\\
\omega_K&=0\\
\u_S&=r_S\i_S+l_S\frac{\d\isab}{\d\tau}\\
&\rotatebox[origin=c]{270}{\laplace}\nonumber\\
\frac{\u_{S0}}{s}&=r_S\I_S\mybr{s}+s l_S\I_S\mybr{s}-l_S \i_{S0}\\
\I_S\mybr{s}&=\frac{\u_{S0}}{s\mybr{r_S+l_S s}}+\frac{l_S \i_{S0}}{r_S+l_S s}\\
&=\u_{S0}\mybr{\frac{1}{r_S s}-\frac{l_S}{r_S\mybr{r_S+l_S s}}}+\frac{l_S \i_{S0}}{r_S+l_S s}\\
&=\frac{\u_{S0}}{r_S}\mybr{\frac{1}{s}-\frac{1}{\mybr{\frac{r_S}{l_S}+ s}}}+\frac{\i_{S0}}{\frac{r_S}{l_S}+ s}\\
&\rotatebox[origin=c]{270}{\Laplace}\nonumber\\
\i_S\mybr{\tau}&=\mybr{\frac{\u_{S0}}{r_S}\mybr{1-\e^{-\frac{r_S}{l_S}\tau}}+\i_{S0}\e^{-\frac{r_S}{l_S}\tau}}\sigma\mybr{\tau}
\end{align}

\section{}
\begin{align}
n_N&=\frac{f_{Nenn}\SI{60}{\second\per\minute}}{p}\\
f_{Nenn}&=\frac{p n_N}{\SI{60}{\second\per\minute}}=\frac{2\cdot\SI{4000}{\rounds\per\minute}}{\SI{60}{\second\per\minute}}=\SI{133.33}{\hertz}\\
T_{Bez}&=\frac{1}{2\pi f_{Nenn}}=\frac{1}{2\pi\SI{133.33}{\hertz}}=\SI{1.1937}{\milli\second}\\
I_3\mybr{t}&=I_{Bez}\Re\mybr{\i_S\mybr{\frac{t}{T_{Bez}}}\e^{-\j\SI{240}{\degree}}}\\
&=\SI{7.071}{\ampere}\cdot\Re\mybr{\mybr{\frac{\num{0.1}\e^{\j\SI{240}{\degree}}}{r_S}\mybr{1-\e^{-\frac{r_S}{l_S}\frac{t}{T_{Bez}}}}+\num{0.65}\e^{\j\SI{20}{\degree}}\e^{-\frac{r_S}{l_S}\frac{t}{T_{Bez}}}}\sigma\mybr{\frac{t}{T_{Bez}}}\e^{-\j\SI{240}{\degree}}}\\
&=\SI{7.071}{\ampere}\cdot\Re\mybr{\mybr{\frac{\num{0.1}}{r_S}\mybr{1-\e^{-\frac{r_S}{l_S}\frac{t}{T_{Bez}}}}+\num{0.65}\e^{-\j\SI{220}{\degree}}\e^{-\frac{r_S}{l_S}\frac{t}{T_{Bez}}}}\sigma\mybr{t}}\\
&=\SI{7.071}{\ampere}\mybr{\frac{\num{0.1}}{r_S}\mybr{1-\e^{-\frac{r_S}{l_S}\frac{t}{T_{Bez}}}}-\num{0.4979}\e^{-\frac{r_S}{l_S}\frac{t}{T_{Bez}}}}\sigma\mybr{t}\\
\end{align}
Aus der Messkurve kann man $I_3\mybr{t\rightarrow\infty}=\SI{15}{\ampere}$ und $\tau_{mess}=\SI{10}{\milli\second}$ ablesen.
\begin{figure*}[!hp]
	\centering
	\begin{tikzpicture}
	\begin{axis}[
	xlabel=$t$ in \si{\milli\second},
	ylabel=$I_3$ in \si{\ampere},
	grid,
	ymin=-5,
	xmin=-5,
	ymax=20,
	xmax=70,]
	\addplot[domain=-5:0, blue] {-0.4979*7.071};
	\addplot[domain=0:70, blue] {7.071*(0.1/0.04714*(1-e^(-x/10))-0.4979*e^(-x/10))};
	\addplot[domain=0:10, black] {7.071*(0.1/0.04714/10+0.4979/10)*x-0.4979*7.071};
	\draw [decorate,decoration={brace,amplitude=5pt},xshift=0pt,yshift=4pt]
	(axis cs:0,15) -- (axis cs:10,15) node [black,midway,yshift=10pt] 
	{\footnotesize $\tau_{mess}$};
	\end{axis}
	\end{tikzpicture}
\end{figure*}
\begin{align}
I_3\mybr{t\rightarrow\infty}&=\SI{7.071}{\ampere}\frac{\num{0.1}}{r_S}\\
r_S&=\frac{\SI{7.071}{\ampere}\cdot\num{0.1}}{\SI{15}{\ampere}}=\num{0.04714}\\
\tau_{mess}&=T_{Bez}\frac{l_S}{r_S}\\
l_S&=\frac{\tau_{mess}r_S}{T_{Bez}}=\frac{\SI{10}{\milli\second}\cdot\num{0.04714}}{\SI{1.1937}{\milli\second}}=\num{0.3949}
\end{align}

\clearpage
%%%%%%%%%%%%%%%%%%%%%%%%%%%%%%%%%%%%%%%%%%%%%%%%%%%%%%%%%%%%%%%%%%%%%%%%%%%%%%%%%%%%%%%%%%%%%%%%%%%%
%%%%%%%%%%%%%%%%%%%%%%%%%%%%%%%%%%%%%%%%%%%%%%%%%%%%%%%%%%%%%%%%%%%%%%%%%%%%%%%%%%%%%%%%%%%%%%%%%%%%
\part{2014 APSM}
\section{}
\begin{align}
U_{Bez}&=\sqrt{2}U_{Nenn,Str}=\sqrt{2}\frac{U_N}{\sqrt{3}}=\sqrt{2}\frac{\SI{120}{\volt}}{\sqrt{3}}=\SI{97.98}{\volt}\\
I_{Bez}&=\sqrt{2}I_{Nenn,Str}=\sqrt{2}I_N=\sqrt{2}\cdot\SI{5}{\ampere}=\SI{7.071}{\ampere}\\
n_N&=\frac{f_{Nenn}\SI{60}{\second\per\minute}}{p}\\
f_{Nenn}&=\frac{n_N p}{\SI{60}{\second\per\minute}}=\frac{\SI{3000}{\rounds\per\minute}\cdot 3}{\SI{60}{\second\per\minute}}=\SI{150}{\hertz}\\
T_{Bez}&=\frac{1}{2\pi f_{el,Nenn}}=\frac{1}{2\pi\SI{150}{\hertz}}=\SI{1.061}{\milli\second}\\
M_{Bez}&=\frac{3 U_{Nenn,Str}I_{Nenn,Str}p}{\omega_{el,Bez}}=\frac{3 \frac{U_N}{\sqrt{3}}I_{N}p}{2\pi f_{el,Nenn}}=\frac{3\frac{\SI{120}{\volt}}{\sqrt{3}}\SI{5}{\ampere}\cdot 3}{2\pi\SI{150}{\hertz}}=\SI{3.308}{\newton\metre}\\
R_{Bez}&=\frac{U_{Bez}}{I_{Bez}}=\frac{\SI{97.98}{\volt}}{\SI{7.071}{\ampere}}=\SI{13.86}{\ohm}\\
L_{Bez}&=\frac{U_{Bez}T_{Bez}}{I_{Bez}}=\frac{\SI{97.98}{\volt}\cdot\SI{1.061}{\milli\second}}{\SI{7.071}{\ampere}}=\SI{0.01470}{\henry}
\end{align}

\section{}
\begin{align}
\isab&=\frac{2}{3I_{Bez}}\mybr{I_1+I_2 \e^{\j\SI{120}{\degree}}+I_3\e^{\j\SI{240}{\degree}}}\\
&=\frac{2}{3\cdot\SI{7.071}{\ampere}}\mybr{\SI{-2.828}{\ampere}+\SI{5.657}{\ampere} \e^{\j\SI{120}{\degree}}+\SI{-2.828}{\ampere}\e^{\j\SI{240}{\degree}}}\\
&=\num{-0.4000}+\j\num{0.6928}\\
\isdq&=\isab\e^{-\j\gamma_m}=\mybr{\num{-0.4000}+\j\num{0.6928}}\e^{-\j\SI{10}{\degree}}=\num{-0.2736}+\j\num{0.7517}\\
m_R&=\Psi_M i_q +\mybr{l_d-l_q}i_d i_q = \num{1}\cdot\num{0.7517}+\mybr{\num{0.5}-\num{0.9}}\cdot\mybr{\num{-0.2736}}\cdot\num{0.7517}=\num{0.8340}
\end{align}

\section{}
\begin{align}
m_{R,soll}&=\Psi_M i_q +\underbrace{\mybr{l_d-l_q}}_{0} i_d i_q=\psi_M i_q\\
i_q&=\frac{m_R,soll}{\Psi_M}=\frac{\num{0.7}}{1}=\num{0.7}
\end{align}
\begin{figure*}[!ht]
	\centering
	\begin{tikzpicture}[>=triangle 45,thick,node distance=0.5cm]
	
	\path (0,0) coordinate (origin1);
	\path (1*8,0) coordinate (alpha);
	\path (0,0.7*8) coordinate (beta);
	\path (100:0.7*6) coordinate (isq);
	\path (10:1*6) coordinate (PsiM);
	\path (0,0.7108*6) coordinate (iSBLDC);
	\path (90:0.6*6) coordinate (arc1);
	\path (0:0.8*6) coordinate (arc2);
	
	\draw [->] (origin1) -- (alpha);
	\draw [->] (origin1) -- (beta);
	\draw [->] (origin1) -- node[below left]{$\j i_q$} (isq);
	\draw [->] (origin1) -- node[above]{$\PPsi_M$} (PsiM);
	\draw [->] (origin1) -- node[above right]{$\i_{S,BLDC}$} (iSBLDC);
	\draw (iSBLDC) -- (isq);
	\draw (arc1) arc (90:100:0.6*6) node[below right]{$\gamma_m$};
	\draw (arc2) arc (0:10:0.8*6) node[below left=0.2cm]{$\gamma_m$};
	
	\node [below of =alpha] {$\alpha$};
	\node [right of =beta] {$\beta$};
	
	\end{tikzpicture}
\end{figure*}

Der Strom $\i_{S,BLDC}$ muss wenn man ihn auf die Richtung von $i_q$ projiziert, die gleiche Länge wie $i_q$ besitzen.
\begin{align}
i_{S,BLDC}&=\frac{i_q}{\cos\mybr{\SI{10}{\degree}}}=\frac{\num{0.7}}{\cos\mybr{\SI{10}{\degree}}}=\num{0.7108}\\
\i_{S,BLDC}&=\num{0.7108}\e^{\j\SI{90}{\degree}}\\
I_1&=I_{Bez}\Re\mybr{\i_{S,BLDC}\e^{\j\SI{0}{\degree}}}=\SI{7.071}{\ampere}\cdot\Re\mybr{\num{0.7108}\e^{\j\SI{90}{\degree}}}=\SI{0}{\ampere}\\
I_2&=I_{Bez}\Re\mybr{\i_{S,BLDC}\e^{-\j\SI{120}{\degree}}}=\SI{7.071}{\ampere}\cdot\Re\mybr{\num{0.7108}\e^{\j\SI{90}{\degree}}\e^{-\j\SI{120}{\degree}}}=\SI{4.353}{\ampere}\\	
I_3&=I_{Bez}\Re\mybr{\i_{S,BLDC}\e^{-\j\SI{240}{\degree}}}=\SI{7.071}{\ampere}\cdot\Re\mybr{\num{0.7108}\e^{\j\SI{90}{\degree}}\e^{-\j\SI{240}{\degree}}}=\SI{-4.353}{\ampere}
\end{align}

\section{}
\begin{align}
\u_S&=r_S\i_S+\frac{\d\PPsi_S}{\d\tau}+\j\omega_K\PPsi_S\\
\psdq&=l_d i_d+\Psi_M+\j l_q i_q\\
&=\num{0.5}\mybr{\num{-0.2736}}+1+\j\num{0.9}\cdot\num{0.7517}\\
&=\num{0.8632}+\j\num{0.6765}\\
\psab&=\psdq\e^{\j\gamma_m}=\mybr{\num{0.8632}+\j\num{0.6765}}\e^{\j\SI{10}{\degree}}\\
&=\num{0.7326}+\j\num{0.8161}\\
\PPsi_S&=\Psi_S\e^{\j\omega_m \tau}\\
\omega_m \tau_0&=\gamma_m\\
\frac{\d\PPsi_S}{\d\tau}&=\j\omega_m\Psi_S\e^{\j\omega_m\tau}=\j\omega_m\PPsi_S\\
\usab&=r_S\isab+\frac{\d\psab}{\d\tau}+\j\omega_K\psab\\
\omega_K&=0\\
\usab&=r_S\isab+\j\omega_m\psab\\
&=\num{0.05}\mybr{\num{-0.4000}+\j\num{0.6928}}+\j\num{0.5}\mybr{\num{0.7326}+\j\num{0.8161}}\\
&=\mybr{\num{-0.02000}+\j\num{0.03464}}+\mybr{\num{-0.4081}+\j\num{0.3663}}\\
&=\num{-0.4281}+\j\num{0.4009}
\end{align}
\begin{figure*}[!h]
	\centering
	\begin{tikzpicture}[>=triangle 45,thick,node distance=0.5cm]
	
	\path (0,0) coordinate (origin1);
	\path (1*8,0) coordinate (alpha);
	\path (0,1*8) coordinate (beta);
	\path (-0.4281*10,0.4009*10) coordinate (us);
	\path (-0.4000*10,0.6928*10) coordinate (is);
	\path (0.7326*10,0.81619*10) coordinate (Psis);
	\path (-0.4081*10,0.3663*10) coordinate (omegaPsis);
	
	\draw [->] (origin1) -- (alpha);
	\draw [->] (origin1) -- (beta);
	\draw [->] (origin1) -- node[above right]{$\u_S$} (us);
	\draw [->] (origin1) -- node[above right]{$\i_S$} (is);
	\draw [->] (origin1) -- node[above left]{$\PPsi_S$} (Psis);
	\draw [->] (omegaPsis) -- node[below left]{$r_S\i_S$} (us);
	\draw [->] (origin1) -- node[below left]{$\j\omega_m\Psi_S$} (omegaPsis);
	
	\node [below of =alpha] {$\alpha$};
	\node [right of =beta] {$\beta$};
	
	\end{tikzpicture}
\end{figure*}

\section{}
\subsection{}
Betrachtung im rotorfesten Koordinatensystem, daher gilt $\omega_K=\omega_m$.
\begin{align}
\u_S&=r_S\i_S+\frac{\d\PPsi_S}{\d\tau}+\j\omega_K\PPsi_S\\
0&=r_S\mybr{i_d+\j i_q}+\frac{\d}{\d\tau}\mybr{l_d i_d + \Psi_M + \j l_q i_q}+\j \omega_m \mybr{l_d i_d + \Psi_M + \j l_q i_q}\\
\frac{\d\Psi_M}{\d\tau}&=0\\
0&=r_S i_d+l_d\frac{\d i_d}{\d\tau} -\omega_m l_q i_q\\
0&=r_S i_q+l_q\frac{\d i_q}{\d\tau}+\omega_m l_d i_d + \omega_m\Psi_M \\
\end{align}

\subsection{}
\begin{align}
\frac{\d i_d}{\d\tau}&=\frac{1}{l_d}\mybr{-r_S i_d+\omega_m l_q i_q}\\
&=\frac{1}{\num{0.5}}\mybr{\num{-0.05}\cdot\mybr{\num{-0.2736}}+\num{0.5}\cdot\num{0.9}\cdot\num{0.7517}}\\
&=\num{0.7039}\\
\frac{\d i_q}{\d\tau}&=\frac{1}{l_q}\mybr{-r_S i_q-\omega_m l_d i_d -\omega_m \Psi_M}\\
&=\frac{1}{\num{0.9}}\mybr{\num{-0.05}\cdot\num{0.7517}-\num{0.5}\cdot\num{0.5}\cdot\mybr{\num{-0.2736}}-\num{0.5}\cdot 1}\\
&=\num{-0.5213}
\end{align}

\subsection{}
\begin{align}
\frac{\d}{\d\tau} &\rightarrow 0\\
0&=r_S i_d -\omega_m l_q i_q\\
i_d&=\frac{\omega_m l_q}{r_S} i_q\\
0&=r_S i_q+\omega_m l_d i_d + \omega_m\Psi_M \\
0&=r_S i_q+\omega_m l_d \frac{\omega_m l_q}{r_S} i_q + \omega_m\Psi_M \\
i_q&=-\frac{\omega_m \Psi_M}{r_S+\omega_m^2\frac{l_d l_q}{r_S}}=-\frac{\num{0.5}\cdot 1}{\num{0.05}+\num{0.5}^2\frac{\num{0.5}\cdot\num{0.9}}{\num{0.05}}}=\num{-0.2174}\\
i_d&=\frac{\omega_m l_q}{r_S} i_q\\
i_d&=\frac{\num{0.5}\cdot\num{0.9}}{\num{0.05}}\mybr{\num{-0.2174}}=\num{-1.957}\\
m_R&=\Psi_M i_q+\mybr{l_d-l_q}i_d i_q\\
&=1\cdot\mybr{\num{-0.2174}}+\mybr{\num{0.5}-\num{0.9}}\cdot\mybr{\num{-1.957}}\cdot\mybr{{\num{-0.2174}}}=\num{-0.3876}
\end{align}

%\fi




\end{document}
